\documentclass{scrartcl}

\usepackage{graphicx}
\usepackage[utf8]{inputenc}
\usepackage[T1]{fontenc}
\usepackage{lmodern}
\usepackage[english]{babel}
\usepackage{amsmath}
\usepackage{amsthm}
\usepackage{mathtools}
\usepackage{amssymb}
\usepackage{listings}
\usepackage{xparse}
\usepackage{geometry}
\usepackage{enumerate}
\usepackage{tikz}
\usepackage[style=english]{csquotes}
\usepackage[language=english, backend=biber, style=alphabetic, sorting=nyt]{biblatex}

\usetikzlibrary{babel, positioning, shapes.geometric, arrows, arrows.meta}
\addbibresource{bibliography.bib}

\title{Some Notes on Morphisms, Regular Maps and Elliptic Curves}
\author{Simon Pohmann}

\newcommand{\N}{\mathbb{N}}
\newcommand{\Z}{\mathbb{Z}}
\newcommand{\F}{\mathbb{F}}
\newcommand{\I}{\mathbb{I}}
\newcommand{\V}{\mathbb{V}}
\newcommand{\proj}{\mathrm{proj}}
\newcommand{\Quot}{\mathrm{Quot}}
\renewcommand{\O}{O}

\newcommand\restr[2]{{
    \left.\kern-\nulldelimiterspace
    #1
    \vphantom{\big|}
    \right|_{#2}
}}

\newtheorem{prop}{Proposition}[section]
\newtheorem{theorem}[prop]{Theorem}
\newtheorem{lemma}[prop]{Lemma}
\newtheorem{corollary}[prop]{Corollary}

\theoremstyle{definition}
\newtheorem{problem}[prop]{Problem}
\newtheorem{alg}[prop]{Algorithm}
\newtheorem{definition}[prop]{Definition}
\newtheorem{example}[prop]{Example}
\newtheorem{remark}[prop]{Remark}

\begin{document}

\maketitle

\tableofcontents

\section{Projective morphisms}
Let $V \subseteq \mathbb{P}^n, W \subseteq \mathbb{P}^m$ be projective varieties.

\begin{definition}
    The affine coordinate ring of $V$ is defined as 
    \begin{equation*}
        k[V] := k[x_1, ..., x_n]/\mathbb{I}_{\mathrm{aff}}(V \cap \mathbb{A}^n)
    \end{equation*}
    The projective coordinate ring of $V$ is defined as
    \begin{equation*}
        S(V) := S_k(V) := k[x_0, ..., x_n]/\mathbb{I}_{\mathrm{proj}}(V)
    \end{equation*}
\end{definition}
Now we want to build maps by interpreting elements of these rings as polynomials.
This is quite subtle, as it is hard to get a well-defined notion.
\begin{definition}
    Let $u_0, ..., u_m \in S(V)$ be homogeneous of same degree.
    For a point $a = [a_0 : ... : a_n] \in V$ say that $[u_0 : ... : u_m]$ is defined on $a$ if there are polynomials $f_0, ..., f_m \in k[x_0, ..., x_n]$ homogeneous of same degree with
    \begin{equation*}
        (f_0(a_0, ..., a_n), ..., f_m(a_0, ..., a_n)) \neq (0, ..., 0) \quad \text{and} \quad \forall i: \ \overline{f}_i = u_i \in S(V)
    \end{equation*}
    In this case, define the value of $[u_0 : ... : u_m]$ at $a$ as
    \begin{equation*}
        [u_0 : ... : u_m](a) := [f_0(a_0, ..., a_n) : ... : f_m(a_0, ..., a_n)] \in \mathbb{P}^m
    \end{equation*}
\end{definition}
\begin{lemma}
    This is well-defined.
\end{lemma}
Using this, define the notion of rational map and morphism.
\begin{definition}
    A partial map $\phi: V \to W$ is called rational, if there are $u_0, ..., u_m \in S(V)$ homogeneous of same degree such that for all points $a \in V$ at which $\phi$ is defined, have
    \begin{equation*}
        [u_0 : ... : u_m](a) \ \text{is defined and} \ [u_0 : ... : u_m](a) = \phi(a)
    \end{equation*}
    Furthermore, $\phi$ is called morphism if it is defined on the whole of $V$.
    We will also denote the morphism by $\phi = [u_0 : ... : u_m]$.
\end{definition}
The condition ``homogeneous of same degree'' is quite bulky, and moreover, the representation $[u_0 : ... : u_m]$ is not unique.
Concretely, a morphism determines the tuple $(u_0, ..., u_m)$ only up to scaling by $S(V) \setminus \{0\}$.
Using the affine coordinate ring, we find a much nicer characterization.
\begin{theorem}[Characterization using the coordinate rings]
    Consider the $S(V)$-module
    \begin{equation*}
        M^{(m)} := \{ (u_0, ..., u_m) \in S(V)^{m + 1} \ | \ u_0, ..., u_m \ \text{homogeneous of same degree} \}
    \end{equation*}
    Then there is a bijection
    \begin{align*}
        \Phi: \ &(M^{(m)} \setminus \{0\}) \ / \ (S(V) \setminus \{0\}) \to \Quot(k[V])^m, \\
        &\overline{(u_0, ..., u_m)} \mapsto \left( \frac {u_i(1, x_1, ..., x_n)} {u_j(1, x_1, ..., x_n)} \right)_{i \neq j} \text{where $u_j(1, x_1, ..., x_n) \neq 0 \in k[V]$}
    \end{align*}
    that is compatible with evaluation maps, i.e. for $u_1, ..., u_m \in k[V]$ and $v_1, ..., v_r \in k[W]$ such that $g(u_0, ..., u_m) = 0$ for all $g \in \I(W)$ homogeneous, have
    \begin{equation*}
        \Phi\left(\mathrm{ev}_{(u_1, ..., u_m)}((v_1, ..., v_r))\right) = \mathrm{ev}_{\Phi((v_1, ..., v_r))}\left(\Phi(u_1, ..., u_m)\right)
    \end{equation*}
    Further we can express evaluation of $(u_0, ..., u_m) \in M^{(m)}$ at a point $a = [1 : a_1 : ... : a_n] \in V$ for which $[u_0 : ... : u_m](a)$ is defined as follows:
    There exist elements $g, f_1, ..., f_m \in k[V]$ with $\Phi((f_1/g, ..., f_m/g)) = (u_0, ..., u_m)$ such that one of the following is the case:
    \begin{itemize}
        \item Either $g(a_1, ..., a_m) \neq 0$ and
        \begin{equation*}
            [1 : f_1(a)/g(a) : ... : f_m(a)/g(a)] = [u_0 : ... : u_m](a)
        \end{equation*}
        \item Or $g(a_1, ..., a_m) = 0, \ (f_1(a), ..., f_m(a)) \neq (0, ..., 0)$ and $[u_0 : ... : u_m](a)$ is a point in the hyperplane at infinity
    \end{itemize}
    Because of this, the field $k(V) := \Quot(k[V])$ is also called the function field of $V$. Note that up to isomorphism, it does not depend on the choice of the embedding $\mathbb{A}^n \to \mathbb{P}^n$.
\end{theorem}
In other words, we have a nice 1-1 correspondence between rational maps $V \to \mathbb{P}^m$ with maximal set of definition to tuples in $k(V)^m$.
From now on, we will not distinguish between those in many cases, and also use the notation $[u_1, ..., u_m]$ with elements $u_i \in k(V)$ to denote the rational map given by $\Phi^{-1}((u_1, ..., u_m))$.

The affine characterization just has the problem that it gives no nice definition of the value at infinity.
In the end, computing this will always require homogenizing the polynomials.

\section{Regular Maps}
Now we want to examine a very similar concept, namely regular maps.
Honestly, I am still not completely sure how exactly they differ, but hopefully, we will work it out now. 

\begin{definition}[Regular map]
    Let $X$ be an affine or projective variety. A map $f: U \to k$ for $U \subseteq X$ is called regular at a point $p \in U^\circ$, if there exists an open neighborhood $W \subseteq U$ of $p$ such that $f: W \to k \subseteq \mathbb{P}^1 \ \text{resp.} \ \mathbb{A}^1$ is a rational map (in the projective sense) and defined on the whole of $W$.
    
    If $U$ is open, call $f$ regular, if $f$ is regular at all $p \in U$. Write $\mathcal{O}_X(U)$ for the $k$-algebra of regular functions on $U$.
\end{definition}
\begin{lemma}
    The notions of regular map on affine and projective varieties are compatible.
    Concretely, let $X \subseteq \mathbb{P}^n$ be a projective variety and $X_a = X \cap \mathbb{A}^n \subseteq \mathbb{A}^n \subseteq \mathbb{P}^n$ an affine chart.
    Then a map $f: X_a \to k$ is regular at $p \in X_a$ in the sense of projective varieties iff it is regular at $p$ in the sense of affine varieties.
\end{lemma}
\begin{proof}
Trivial.
\end{proof}
\begin{definition}[Function germ]
    Let $X$ be an affine or projective variety and $p \in X$.
    The elements of
    \begin{equation*}
        \mathcal{O}_{X, p} := \{ (U, f) \ | \ U \subseteq X \ \text{open}, f: U \to k \ \text{regular} \} / \sim
    \end{equation*}
    where
    \begin{equation*}
        (U, f) \sim (U', f') :\Leftrightarrow \restr{f}{W} = \restr{f'}{W} \ \text{for some $W \subseteq U \cap U'$ open}
    \end{equation*}
    are called function germs at $p$.
    Note that $\mathcal{O}_{X, p}$ is again a $k$-algebra.
\end{definition}
The function germs allow a nice characterization using again the coordinate ring.
\begin{prop}
    Let $X \subseteq \mathcal{A}^n$ be an affine variety. At $p \in X$ have
    \begin{equation*}
        \mathcal{O}_{X, p} \cong k[X]_{\mathfrak{m}_p}
    \end{equation*} 
\end{prop}
It really seems as though there is a lot of technical complexity, but not fundamental one.
Probably, it will get simpler when considering quasi-projective varieties, as we do not require the distinction between affine and projective varieties anymore.
Further, one difference between regular maps and (projective) morphisms is that the former is usually defined for affine varieties, while the latter is defined for projective varieties.
However, morphisms of affine varieties differ somewhat from regular maps, as it is possible to define by a global condition instead of locally.

Finally, for quasi-projective varieties, the notion of regular map and morphism are exactly the same.
Hence, any difference is just in the domain and range, which usually are not varieties in the case of regular maps.
\end{document}