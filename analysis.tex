
\subsection{Young's inequality}
\begin{equation}
    xy \leq \frac {x^p} p + \frac {y^q} q \text{ for } \frac 1 p + \frac 1 q = 1, \ x, y \geq 0 \nonumber
\end{equation}
\paragraph{Proof} By convexity of $\log$, have
\begin{equation}
    \begin{split}
        &\frac 1 p \log x^p + \frac 1 q \log y^q \leq \log \left( \frac 1 p x^p + \frac 1 q y^q \right) \\
        \Rightarrow \ &\log ( x y ) \leq \log \left( \frac 1 p x^p + \frac 1 q y^q \right) \nonumber
    \end{split}
\end{equation}

\subsection{Hölder's inequality} 
For measurable functions $f, g$ and $\frac 1 p + \frac 1 q = 1$ (w.r.t measure $\mu$) have:
\label{hoelder_inequality}
\begin{equation}
    \| fg \|_1 = \int | fg | d\mu \leq \left( \int |f|^p d\mu \right)^{\frac 1 p} \left( \int |g|^q d\mu \right)^{\frac 1 q} = \|f\|_p \|g\|_q \nonumber
\end{equation}
\paragraph{Proof} By Young's inequality have
\begin{equation}
    \begin{split}
        &\frac {|fg|} {\| f \|_p \| g \|_q} \leq \frac {|f|^p} {p\|f\|_p^p} + \frac {|g|^q} {q\|f\|_q^q} \\
        \Rightarrow \ &\frac {|fg|} {\| f \|_p \| g \|_q} \leq \frac 1 {p \|f\|_p^p} \|f\|_p^p + \frac 1 {q \|g\|_q^q} \|f\|_q^q = 1 \nonumber
    \end{split}
\end{equation}

\subsection{Partial Summation formula}
Let $(a_n)_n$ be a sequence in $\C$ and $f: \R \to \R$ a continuously differentiable function.
Let $A(t) := \sum_{n \leq t} a_n$. Then
\begin{equation*}
    \sum_{x < n \leq y} a_n f(n) = A(y)f(y) - A(x)f(x) - \int_x^y A(t)f'(t)dt
\end{equation*}
\paragraph{Proof idea} With $n_1 := \lfloor x \rfloor + 1, n_2 := \lfloor y \rfloor$ have
\begin{align*}
    \sum_{x < n \leq y} a_n f(n) &= \sum_{n_1 \leq n \leq n_2} f(n)(A(n) - A(n - 1)) \\
    &= \sum_{n_1 \leq n \leq n_2} A(n) f(n) - \sum_{n_1 - 1 \leq n \leq n_2 - 1} A(n) f(n + 1) \\
    &= A(n_2) f(n_2) - A(n_1 - 1) f(n_1) + \sum_{n_1 \leq n \leq n_2 - 1} A(n) \underbrace{(f(n) - f(n + 1))}_{= -\int_n^{n+1} f'(t) dt} \\
    &= A(n_2) f(n_2) - A(n_1 - 1) f(n_1) - \int_{n_1}^{n_2} A(t) f'(t) dt
\end{align*}
Using further that $A(n_2) f(n_2) = A(y) f(y) - \int_y^{n_2} A(t) f'(t) dt$ and $A(n_1) f(n_1) = A(x) f(x) + \int_{n_1}^x A(t) f'(t) dt$ yields the claim.$\hfill\square$

\subsection{Fourier transform}
\begin{itemize}
    \item For $f \in S(\R) := \{ f: \R \to \C \ | \ f \ \text{$C^\infty$-function}, \ \forall k \in \N_{> 0}: f^{(i)}(x) = O(|x|^{-k}) \}$ have
    \begin{equation*}
        \hat{f}(\xi) := \int_{-\infty}^{\infty} f(x) \exp(-2\pi i x \xi) dx \in S(\R)
    \end{equation*}
    and
    \begin{equation*}
        f(x) = \int_{-\infty}^{\infty} \hat{f}(\xi) \exp(2\pi i \xi x) d\xi
    \end{equation*}
    \item For $f \in S(\R/\Z) = \{ f: \R / \Z \to \C \ | \ \text{$f$ is $C^\infty$-function}\}$ have
    \begin{equation*}
        \hat{f}(n) := \int_0^1 f(x) \exp(-2\pi i x n) dx \in S(\Z)
    \end{equation*}
    where $S(\Z) := \{ g: \Z \to \C \ | \ \forall k \in \N_{> 0}: g(n) = O(n^{-k})\}$ and
    \begin{equation*}
        f(x) = \sum_{n \in \Z} \hat{f}(n) \exp(2\pi i n x)
    \end{equation*}
    \item More generally, for
    \begin{equation*}
        f \in S(\R^n) := \{ f: \R^n \to \C \ | \ \text{$f$ is $C^\infty$-function}, \ \forall k \in \N_{> 0}: f(x) = O(\|x\|^{-k})\}
    \end{equation*}
    have
    \begin{equation*}
        \hat{f}(\xi) := \int_{\R^n} f(x) \exp(-2\pi i \langle x, \xi \rangle) dx \in S(\R^n)
    \end{equation*}
    and
    \begin{equation*}
        f(x) = \int_{\R^n} \hat{f}(\xi) \exp(2\pi i \langle x, \xi \rangle) d\xi
    \end{equation*}
    \item For a lattice $L \subseteq \R^n$ and
    \begin{equation*}
        f \in S(\R^n / L) := \{ f: \R^n / L \to \C \ | \ \text{$f$ is $C^\infty$-function} \}
    \end{equation*}
    have
    \begin{equation*}
        \hat{f}(\xi) = \int_{\text{Fundamental parallelepiped of $L$}} f(x) \exp(-2\pi i \langle x, \xi \rangle) dx \in S(L^*)
    \end{equation*}
    where $S(L^*) := \{ f: L^* \to \C \ | \ \forall k \in \N_{> 0}: f(x) = O(\|x\|^{-k}) \}$ and
    \begin{equation*}
        f(x) = \sum_{\xi \in L^*} \hat{f}(\xi) \exp(2\pi i \langle x, \xi \rangle)
    \end{equation*}
\end{itemize}

\subsection{Poisson Summation Formula}
\begin{itemize}
    \item For $f \in S(\R)$ have
    \begin{equation*}
        \sum_{n \in \Z} f(n) = \sum_{n \in \Z} \hat{f}(n)
    \end{equation*}
    \item More generally, for a lattice $L \subseteq \R^n$, any $y \in \R^n$ and $f \in S(\R^n)$ have
    \begin{equation*}
        \sum_{x \in L} f(y + x) = \det(L^*) \sum_{x \in L^*} \hat{f}(x) \exp(2\pi i \langle y, x \rangle)
    \end{equation*}
\end{itemize}
\paragraph{Proof} Consider the functions
\begin{align*}
    F: \R^n \to \C, \quad &y \mapsto \sum_{x \in L} f(y + x) \\
    G: \R^n \to \C, \quad &y \mapsto \det(L^*) \sum_{x \in L^*} \hat{f}(x)\exp(2\pi i\langle y, x \rangle)
\end{align*}
which are obviously periodic on $L$, and hence $F, G \in S(\R^n / L)$.
We show that $\hat{F} = \hat{G} \in S(L^*)$, and then it follows that $F = G$. For the fundamental parallelepiped $P$ of $L$ have
\begin{align*}
    \hat{G}(\xi) &= \int_P G(y) \exp(-2\pi i \langle \xi, y \rangle) dy = \det(L^*) \int_P \ \sum_{x \in L^*} \hat{f}(x) \exp(2\pi i \langle y, x - \xi \rangle) dy \\
    &= \det(L^*) \sum_{x \in L^*} \hat{f}(x) \underbrace{\int_P \exp(2\pi i \langle y, x - \xi \rangle) dy}_{= \begin{cases}
        \vol(P) & \text{if $x = \xi$} \\
        0 & \text{otherwise}
    \end{cases}} = \det(P)\det(L^*) \hat{f}(\xi) = \hat{f}(\xi)
\end{align*}
and by using that $\langle L^*, L \rangle \subseteq \Z$ we also find
\begin{align*}
    \hat{F}(\xi) &= \int_P F(y) \exp(-2\pi i \langle \xi, y \rangle) dy = \int_P \ \sum_{x \in L} f(y + x) \exp(-2\pi i \langle \xi, y \rangle) dy \\
    &= \sum_{x \in L} \ \int_P f(y + x) \exp(-2\pi i \langle \xi, y \rangle) dy = \sum_{x \in L} \ \int_{P - x} f(y) \exp(-2\pi i \langle \xi, y - x \rangle) dy \\
    &= \sum_{x \in L} \ \int_{P - x} f(y) \exp(-2\pi i \langle \xi, y \rangle) dy = \int_{\R^n} f(y)\exp(-2\pi i \langle \xi, y \rangle) dy = \hat{f}(\xi)
\end{align*}
So $F = G$ and clearly $F(y) = G(y)$ for the given $y$.$\hfill\square$

\subsection{The Gaussian Function}
Let $f(x) := \exp(-\pi x^2)$. Then $f \in S(\R)$ with $\hat{f}(\xi) = f(\xi)$ for all $\xi \in \R$.
Further $\int_{-\infty}^\infty f(x) dx = 1$.

\subsection{The Gamma Function}
Define
\begin{equation*}
    \Gamma(s) := \int_0^\infty x^{s - 1} e^{-x} dx \quad \text{for $\Re(s) > 0$}
\end{equation*}
Then for $\Re(s) > 0$ have $\Gamma(s) = \Gamma(s + 1) / s$ and so we find a meromorphic continuation to $\C$ with poles only at $\Z_{< 0}$.

\subsection{Euler Reflection Formula}
For all $s \in \C$ have
\begin{equation*}
    \Gamma(s)\Gamma(s - 1) = \frac \pi {\sin(\pi s)}
\end{equation*}
(if $s \in \Z$, this means that both sides have poles of equal order at $s$)

In particular, $\Gamma(s)$ has no zeros on $\C$.

\subsection{Perron's Formula}
Let $2 \leq T \leq 2x$ and $c := 1 + \frac 1 {\log x}$. Consider a sequence $(a_n)_n$ in $\C$ with $|a_n| \leq (\log n)^2$. Then
\begin{equation*}
    \sum_{n < x} a_n = \frac 1 {2\pi i} \int_{c - iT}^{c + iT} \frac {x^s} s \left( \sum_{n \geq 1} \frac {a_n} {n^s} \right) ds + O\left(\frac {x(\log x)^3} T\right)
\end{equation*}
\paragraph{Proof idea} Use the ``weak Perron's formula'' to distinguish whether $n < x$ or $n > x$.
\subsubsection*{Lemma}
Let $y > 0, y \neq 1$. Then for all $c > 0$ and $T \geq 2$ have
\begin{equation*}
    \frac 1 {2\pi i} \int_{c - iT}^{c + iT} \frac {y^s} s ds = O\left( \frac {y^c} {T |\log y|} \right) + \begin{cases}
        1 & \text{if $y > 1$} \\
        0 & \text{if $y < 1$}
    \end{cases}
\end{equation*}

\subsection{Technical lemmas on holomorphic functions}
\label{bounds_complex_analytic_functions}
Let $f(z)$ be an analytic function on $|z| \leq R$ with Taylor series $f(z) = \sum c_n z^n$. Then
\begin{equation*}
    |c_n| \leq \frac {8 \max_{|z| = R} \Re(f(z) - f(0))} {R^n}
\end{equation*}
Further, let $f(z)$ be an analytic function on $|z| \leq R$ with $f(0) \neq 0$.
Let $z_1, ..., z_k$ be the zeros of $f$ within $|z| \leq \frac R 2$ counted with multiplicity.
Then there is $\epsilon > 0$ such that for $|z| \leq (\frac 1 2 - \epsilon)R$ have
\begin{equation*}
    \frac {f'(z)} {f(z)} = \sum_j \frac 1 {z - z_j} + O\left( \frac 1 R \max_{|x| = R} \log\left| \frac {f(x)} {f(0)} \right| \right)
\end{equation*}
Further, let $f(z)$ be an analytic function on $|z| \leq R$.
Then the number of zeros on $|z| \leq \frac R 2$ (counted with multiplicity) is bounded by $2\max_{|x| = R} \log\left| f(x)/f(0) \right|$.
