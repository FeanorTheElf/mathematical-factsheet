\documentclass{scrartcl}

\usepackage{graphicx}
\usepackage[utf8]{inputenc}
\usepackage[T1]{fontenc}
\usepackage{lmodern}
\usepackage[ngerman]{babel}
\usepackage{amsmath}
\usepackage{mathtools}
\usepackage{amssymb}
\usepackage{listings}
\usepackage{xparse}
\usepackage{geometry}
%\geometry{a4paper,margin=1in}

\title{Collection of arbitrary mathematical facts}
\date{}

\newcommand{\Ex}{\mathrm{E}}
\newcommand{\R}{\mathbb{R}}
\newcommand{\N}{\mathbb{N}}
\newcommand{\Z}{\mathbb{Z}}
\newcommand{\Q}{\mathbb{Q}}
\newcommand{\C}{\mathbb{C}}
\newcommand{\vol}{\mathrm{vol}}
\newcommand\restr[2]{{
    \left.\kern-\nulldelimiterspace
    #1
    \vphantom{\big|}
    \right|_{#2}
}}

\begin{document}

\maketitle

\tableofcontents

\paragraph{An undeniable fact:} It holds $0 \in \N$. If you do not see that this is obviously, inarguably true, then you are lost.

\section{Set Theory}

\subsection{Zorn's Lemma}
Let $X$ be a partially ordered set, in which every chain has an upper bound. Then $X$ has a maximal element.
\paragraph{Proof} Show that the set $\mathcal{X} \subseteq 2^X$ of chains in $X$ has a maximal element, so $X$ has a maximal chain (whose upper bound then is the required maximal element).

Let $f: 2^X \setminus \{\emptyset\} \to X$ be a choice function for $X$, so $f(S) \in S$ for each $S \subseteq X$. Then define
\begin{equation}
    g: \mathcal{X} \to \mathcal{X}, \quad C \mapsto \begin{cases}
        C, & \text{if $C$ maximal} \\
        C \cup \left\{ f\left(\{ x \in X \ | \ \text{$x$ comparable with $C$} \}\right) \right\}, & \text{otherwise}
    \end{cases} \nonumber
\end{equation}
where we say that an element $x \in X$ is comparable with a set $S \subseteq X$, if $x$ is comparable with $s$ for all $s \in S$.
\subparagraph{Definition Tower} Call a subset $\mathcal{T} \subseteq \mathcal{X}$ tower, if
\begin{itemize}
    \item $\emptyset \in \mathcal{T}$
    \item If $C \in \mathcal{T}$, then $g(C) \in \mathcal{T}$
    \item If $\mathcal{S} \subseteq \mathcal{T}$ is a chain, then $\bigcup \mathcal{S} \in \mathcal{T}$
\end{itemize}
The intersection of towers is a tower, so have a smallest tower $\mathcal{R} := \bigcap \{ \mathcal{T} \subseteq \mathcal{X} \ | \ \text{$\mathcal{T}$ tower} \}$ in $\mathcal{X}$. We show that $\mathcal{R}$ is a chain. Consider the set $\mathcal{C} := \{ A \in \mathcal{R} \ | \ \text{$A$ comparable to $\mathcal{R}$}\}$ of comparable elements in $\mathcal{R}$.

\subparagraph{Show} $\mathcal{C}$ is a tower, so $\mathcal{R} = \mathcal{C}$ and therefore, $\mathcal{R}$ is a chain.

Trivially, we have $\emptyset \in \mathcal{C}$ as $\emptyset \subseteq A$ for each $A \in \mathcal{R}$. For a chain $\mathcal{S} \subseteq \mathcal{C}$ and any $A \in \mathcal{R}$, have either $A \subseteq S$ for some $S \in \mathcal{S}$, so $A \subseteq \bigcup \mathcal{S}$, or $S \subseteq A$ for each $S \in \mathcal{S}$, so $\bigcup \mathcal{S} \subseteq A$. 
Therefore, it is left to show that for $\mathcal{C}$ is closed under $g$. Let $B \in \mathcal{C}$.

\subparagraph{Show} The set $\mathcal{U} := \{ A \in \mathcal{R} \ | \ A \subseteq B \vee g(B) \subseteq A \} \subseteq \mathcal{R}$ is a tower. It then follows that $\mathcal{R} = \mathcal{U}$, so for each $A \in \mathcal{R}$, have $A \subseteq B \subseteq g(B)$ or $g(B) \subseteq A$. Hence, $g(B)$ is comparable to $\mathcal{R}$. Obviously, $\emptyset \in \mathcal{U}$ and for a chain $\mathcal{S} \subset \mathcal{U}$, also $\bigcup \mathcal{S} \in \mathcal{U}$. Additionally, for $U \in \mathcal{U}$, have:

If $g(B) \subseteq U$, then also $g(B) \subseteq g(U)$. 

Otherwise, $U \subseteq B$. If $B = U$, then $g(B) \subseteq g(U)$, so we may assume $U \subsetneq B$. We have that $U \in \mathcal{R}$, so $g(U) \in \mathcal{R}$ (because $\mathcal{R}$ is a tower) and therefore, $B$ is comparable to $g(U)$.
\\$\Rightarrow g(U) \subseteq B$, because if $B \subsetneq g(U)$, we would have $U \subsetneq B \subsetneq g(U)$, however, $g(U) \setminus U$ has at most one element. Hence, $g(U) \in \mathcal{U}$, so $\mathcal{U} = \mathcal{C} = \mathcal{R}$ are towers.

\subparagraph{Show} The set $C := \bigcup \mathcal{R}$ is a maximal element in $\mathcal{X}$.

$\mathcal{R}$ is a chain and a tower, so $C \in \mathcal{R}$. We also have $g(C) \in \mathcal{R}$, as $\mathcal{R}$ is a tower.
\\$\Rightarrow g(C) \subseteq C$ and therefore $C = g(C)$, so $C$ is maximal in $\mathcal{X}$ by definition of $g$.

\subsection{Ultrafilter Lemma}
For each filter $\mathcal{F}$ on a set $X$ there is a ultrafilter $\mathcal{U}$ such that $\mathcal{F} \subseteq \mathcal{U}$. 

\subsection{Product Cardinality}
For infinite set $X$ have $\mathrm{card}(X) = \mathrm{card}(X \times X)$.
\paragraph{Proof idea} Proof the statement for all cardinals $\aleph_\alpha$ by transfinite induction on $\alpha$ to show that $\mathrm{ord}(\aleph_\alpha \times \aleph_\alpha) = \aleph_\alpha$ (using some defined well-order on $\aleph_\alpha \times \aleph_\alpha$).

If $\alpha = \beta + 1$ then show that for each $\mu < \aleph_\alpha$ have $\mathrm{card}(\mu \times \mu) < \aleph_\alpha$ by transfinite induction. For the limit ordinal case, use that $\mathrm{card}(\mu \times \mu) < \aleph_\alpha$ iff $\mathrm{card}(\mu \times \mu) \leq \aleph_\beta$ and $\mathrm{card}(\aleph_\beta \times \aleph_\beta) = \aleph_\beta$.
It follows that then $\mathrm{ord}(\mu \times \mu) < \aleph_\alpha$, so $\mathrm{ord}(\aleph_\alpha \times \aleph_\alpha) = \bigcup_{\mu < \aleph_\alpha} \mathrm{ord}(\mu \times \mu) \leq \aleph_\alpha$.

If $\alpha$ is a limit ordinal, then $\mathrm{ord}(\aleph_\alpha \times \aleph_\alpha) = \bigcup_{\beta < \alpha} \mathrm{ord}(\aleph_\beta \times \aleph_\beta) = \bigcup_{\beta < \alpha} \aleph_\beta = \aleph_\alpha$.

\section{Algebra}

\subsection{Cauchy-Schwarz}

For $x, y \in V$ inner product space, have
\begin{equation}
    \langle x, y \rangle^2 \leq \langle x, x \rangle \langle y, y \rangle \nonumber
\end{equation}
\paragraph{Proof idea} Start with
\begin{equation}
    \langle x, x \rangle \left\langle y - \frac {\langle x, y \rangle}{\langle x, x \rangle} x, \ y - \frac {\langle x, y \rangle}{\langle x, x \rangle} x \right\rangle \geq 0 \nonumber
\end{equation}

\subsection{Sylow Theorems}

For a finite group $G$ with $|G| = n = p^em, \ p \in \mathbb{P}, \ p \perp m$ have:
\begin{itemize}
    \item There is $U \leq G$ with $|U| = p^e$
    \item For $U, V \leq G$ with $|U| = |V| = p^e$ have $U = gVg^{-1}$ for $g \in G$
    \item Let $s$ be the count of $U \leq G, \ |U| = p^e$. Then $s | m$ and $s \equiv 1 \mod p$
\end{itemize}

\paragraph{Proof idea} Use group operations, for 1. on $\chi := \{ U \leq G \ | \ |U| = p^e \}$, for 2. on $\chi := \{ gU \ | \ g \in G \}$ and for 3. on $\chi := \{ U \leq G \ | \ |U| = p^e \}$ with conjugation.

\subsection{Mordell's inequality}
Have $\gamma_d \leq \gamma_{d-1}^{(d - 1) / (d - 2)}$. Inductively, it follows $\gamma_d \leq \gamma_k^{(d-1)/(k-1)}$ ($\gamma$ here is Hermite's constant).
\paragraph{Proof} Let $L$ be a $d$-rank lattice for which Hermite's constant is reached, with dual $L^*$ and $x \in L^*$ with $\|x\| = \lambda(L^*)$.
\begin{equation}
    \begin{split}
        \Rightarrow& \ \left(\langle x \rangle ^\perp \cap L\right)^* = \pi_{\langle x \rangle ^\perp} (L^*) \ \Rightarrow \ \vol(L^*) = \|x\| \ \vol\left(\langle x \rangle^\perp \cup L\right)^* \\
        \Rightarrow& \ \sqrt{\gamma_{n-1}}^{1-n} \lambda(L)^{n-1} \leq \vol\left(\langle x \rangle^\perp \cap L\right) = \|x\| \ \vol(L) \leq \sqrt{\gamma_n} \vol(L^*)^{\frac 1 n} \vol(L) \\
        \Rightarrow& \ \sqrt{\gamma_n} \sqrt{\gamma_{n - 1}}^{n - 1} \geq \frac {\lambda(L)^{n-1}} {\vol(L)^{\frac {n-1} n}} = \sqrt{\gamma_n}^{n-1} \ \Rightarrow \ \sqrt{\gamma_n}^{n-2} \geq \sqrt{\gamma_{n-1}}^{n-1}
    \end{split} \nonumber
\end{equation}
where $M^*$ denotes the unique ``dual'' of $M$ in $\langle M \rangle$.

\subsection{Facts about finite rings}
\begin{itemize}
    \item $\mathbb{F}_q^*$ is cyclic for $q = p^n$
\end{itemize}
\paragraph{Proof} By the theorem on finitely generated abelian groups, have
\begin{equation}
    \mathbb{F}_q^* \cong \Z/n_1\Z \times ... \times \Z/n_s\Z \nonumber
\end{equation}
with $n_1 | ... | n_s$. Assume $s > 1$ and $n_1 \neq 1$. Then $n_s < N := |\mathbb{F}_q^*|$. For $x \in \mathbb{F}_q^*$, have therefore that $\mathrm{ord}(x) | n_s$, so $p(x) = 0$ with $p(X) := X^{n_s} - 1$. 
But this is a contradiction, as $p$ is a polynomial of degree $n_s$ with $N > n_s$ roots in the field $\mathbb{F}_q$.
\begin{itemize}
    \item $(\Z/p^\alpha\Z)^*$ is cyclic if $p > 2$ or $\alpha \leq 2$
\end{itemize}
\paragraph{Proof} Use induction over $\alpha$.
\subparagraph{$\alpha = 1$} Follows directly from the previous point, as $\mathbb{F}_p \cong \Z/p\Z$ as rings.
\subparagraph{$\alpha > 1$} Consider the canonical ring homomorphism
\begin{equation}
    \pi: \Z/p^\alpha\Z \to (\Z/p^\alpha\Z) \ / \ ([p^{\alpha - 1}]), \quad x \mapsto [x] \nonumber
\end{equation}
Then the restriction of $\pi$ to $(\Z/p^\alpha\Z)^*$
\begin{equation}
    f: (\Z/p^\alpha\Z)^* \to \left( (\Z/p^\alpha\Z) \ / \ ([p^{\alpha - 1}]) \right)^*, \quad x \mapsto \pi(x) \nonumber
\end{equation}
is a surjective group homomorphism. We have
\begin{equation}
    \mathrm{ker}(f) = \pi^{-1}(\{1\}) = 1 + ([p^{\alpha - 1}]) = \left\{ 1 + k[p^{\alpha - 1}] \ \middle| \ k \in \{0, ..., p - 1\} \right\} \nonumber
\end{equation}
As $[p^{\alpha - 1}]^2 = 0$, have $\mathrm{ker}(f) = \langle 1 + [p^{\alpha - 1}] \rangle$ by the binomial theorem. On the other hand, by the second isomorphism theorem, have
the ring isomorphy $( (\Z/p^\alpha\Z) \ / \ ([p^{\alpha - 1}]) ) \cong \Z/p^{\alpha - 1}\Z$, which is cyclic by the induction hypothesis. Therefore, $G / \mathrm{im}(f) \cong \mathrm{ker}(f)$ yields:
\begin{equation}
    (\Z/p^\alpha\Z)^* / \langle 1 + [p^{\alpha - 1}] \rangle \cong \langle [g] \rangle \ \text{for some} \ g \in (\Z/p^\alpha\Z)^* \nonumber
\end{equation}
Assume now that $(\Z/p^\alpha\Z)^*$ is not cyclic. Then $\mathrm{ord}(g) \neq (p-1)p^{\alpha - 1}$, so $\mathrm{ord}(g) = (p-1)p^{\alpha - 2}$, as $\mathrm{ord}(1 + [p^{\alpha - 1}]) = p$.
If $\alpha = 2$, then $\mathrm{ord}(g) = p - 1 \perp p$, and the Chinese Remainder theorem yields that 
\begin{equation}
    (\Z/p^\alpha\Z)^* \cong \Z/p\Z \times \Z/(p - 1)p^{\alpha - 2}\Z \cong \Z/(p - 1)p^{\alpha - 1}\Z \nonumber
\end{equation}
and we are done. Therefore, let $\alpha > 2$ and $p > 2$ and consider the mapping
\begin{equation}
    \phi: \Z/p\Z \times \Z/(p-1)p^{\alpha - 2}\Z \to (\Z/p^\alpha\Z)^*, \quad (k, n) \mapsto (1 + k[p^{\alpha - 1}])g^n \nonumber
\end{equation}
which is a homomorphism, as $(1 + k[p^{\alpha - 1}])(1 + l[p^{\alpha - 1}]) = 1 + (l + k)[p^{\alpha - 1}]$ and $\mathrm{ord}(g) = (p-1)p^{\alpha - 2}$ and bijective, so an isomorphism. How to continue from here?

\section{Probabilities}

\subsection{Chernoff-Hoeffding}
$X_1, ..., X_n$ independent, $0 \leq X_i \leq 1$. Then
\begin{equation}
    \Pr\left[\sum X_i - \Ex\left[\sum X_i\right] \geq t\right] \leq \exp\left(-2\frac {t^2} n\right) \nonumber
\end{equation}

\section{Analysis}

\subsection{Inequalities}

\paragraph{Young's inequality}
\begin{equation}
    xy \leq \frac {x^p} p + \frac {y^q} q \text{ for } \frac 1 p + \frac 1 q = 1, \ x, y \geq 0 \nonumber
\end{equation}
\paragraph{Proof} By convexity of $\log$, have
\begin{equation}
    \begin{split}
        &\frac 1 p \log x^p + \frac 1 q \log y^q \leq \log \left( \frac 1 p x^p + \frac 1 q y^q \right) \\
        \Rightarrow \ &\log ( x y ) \leq \log \left( \frac 1 p x^p + \frac 1 q y^q \right) \nonumber
    \end{split}
\end{equation}

\paragraph{Hölder's inequality} For measurable functions $f, g$ and $\frac 1 p + \frac 1 q = 1$ (w.r.t measure $\mu$) have:
\label{hoelder_inequality}
\begin{equation}
    \| fg \|_1 = \int | fg | d\mu \leq \left( \int |f|^p d\mu \right)^{\frac 1 p} \left( \int |g|^q d\mu \right)^{\frac 1 q} = \|f\|_p \|g\|_q \nonumber
\end{equation}
\paragraph{Proof} By Young's inequality have
\begin{equation}
    \begin{split}
        &\frac {|fg|} {\| f \|_p \| g \|_q} \leq \frac {|f|^p} {p\|f\|_p^p} + \frac {|g|^q} {q\|f\|_q^q} \\
        \Rightarrow \ &\frac {|fg|} {\| f \|_p \| g \|_q} \leq \frac 1 {p \|f\|_p^p} \|f\|_p^p + \frac 1 {q \|g\|_q^q} \|f\|_q^q = 1 \nonumber
    \end{split}
\end{equation}

\subsection{Transformation}
$\phi: U \to \mathbb{R}^n$ injective. Then
\begin{equation}
    \int_{\phi(U)} f(x) dx = \int_U f(\phi(x)) \ |\det(D\phi)(u)| \ dx \nonumber
\end{equation}

\section{Topology}

\subsection{Separation axioms}
\begin{description}
    \item[T0] for distinct points $x, y$, have either $x \in U, y \notin U$ or $x \notin U, y \in U$ for open $U$
    \item[T1] for distinct points $x, y$ have $x \in U, y \notin U$ and $x \notin V, y \in V$ for open $U, V$ (equivalent to singletons are closed)
    \item[T2] or Hausdorff; points can be separated by open sets
    \item[T3] T1 + points can be separated from closed sets by open sets
    \item[T4] T1 + closed sets can be separated from closed sets by open sets
\end{description}

\subsection{Universal nets}
Every net $(x_i)_{i \in I}$ has a universal subnet.
\paragraph{Proof idea} 
Consider the filter $\mathcal{F} = \{F \subseteq I \ | \ \exists i \in I \ \forall j \in I: \ j \geq i \Rightarrow j \in F\}$ and use ultrafilter $\mathcal{U} \supseteq \mathcal{F}$ as index set.

\subsection{Initial topologies}
$\{ \bigcap_{\alpha \in \mathcal{F}} f_\alpha^{-1}(U_\alpha) \ | \ \mathcal{F} \subseteq \mathcal{A} \ \text{finite}, \ U_\alpha \in \tau_\alpha \}$ is a basis for the initial topology of $f_\alpha: X \to (X_\alpha, \tau_\alpha)$.

\subsection{Characterization of compactness}
The following are equivalent, where $(X, \tau)$ is a topological space
\begin{itemize}
    \item Every open cover of $X$ has a finite subcover
    \item For all $\mathcal{D} \subseteq 2^X$ of nonempty, closed sets with $\bigcap \mathcal{F} \neq \emptyset$ for each finite $\mathcal{F} \subseteq \mathcal{D}$ have that $\bigcap \mathcal{D} \neq \emptyset$
    \item For each chain $\mathcal{C} \subseteq 2^X$ of nonempty, closed sets have $\bigcap \mathcal{C} \neq \emptyset$
    \item Each universal net converges
    \item Each net has a convergent subnet
    \item Each closed $S \subseteq X$ is compact w.r.t the subspace topology
\end{itemize}
\paragraph{Proof} Interesting is only (iii) $\Rightarrow$ (ii). Given $\mathcal{D} \subseteq 2^X$ consider $\mathcal{S} := \{\mathcal{A} \subseteq \mathcal{D} \ | \ \bigcap \mathcal{A} \neq \emptyset \}$. Then by assumption, $\mathcal{S}$ contains all finite sets.
Also, $\mathcal{S}$ is also closed w.r.t monotone unions, as for a chain $\mathcal{C} \subseteq \mathcal{S}$ have that $\{ \bigcap C \ | \ C \in \mathcal{C} \}$ is a chain of nonempty closed sets, so $\bigcap \{ \bigcap C \ | \ C \in \mathcal{C} \} \neq \emptyset$ by assumption. But this is a lower bound for each $C \in \mathcal{C}$, so for $\bigcup \mathcal{C}$. Therefore, $\bigcup \mathcal{C} \in \mathcal{S}$.

Assume $\mathcal{A} \subseteq 2^\mathcal{D}$ is a set of smallest cardinality $\kappa$ not in $\mathcal{S}$. Then we can well-order $\mathcal{A} = \{a_\xi \ | \ \xi \in \kappa\}$ and get $\mathcal{A} = \bigcup_{\chi \in \kappa} \{ a_\xi \ | \ \xi \in \chi\}$ as $\kappa$ is infinite, so a limit ordinal. Therefore $\mathcal{A}$ is a monotone union of sets in $\mathcal{S}$ (by minimality of $\kappa$), so in $\mathcal{S}$.
Then $\mathcal{S} = 2^\mathcal{D}$ so $\mathcal{D} \in \mathcal{S}$ and therefore $\bigcap \mathcal{D} \neq \emptyset$.

\subsection{Tychonoffs Theorem}
For a collection of compact topological spaces $(X_i)_{i \in I}$ the product space $\prod_{i \in I} X_i$ is compact.
\paragraph{Proof idea} Follows directly from the fact that projections of universal nets are universal, and a space is compact iff every universal net converges.

\subsection{Urysohn's Lemma}
For closed $C_0, C_1$ in a T4 space $X$ there is a continuous $f: X \to [0, 1]$ with $\restr{f}{C_0} = 0$ and $\restr{f}{C_1} = 1$.
\paragraph{Proof idea} Construct by induction open sets $U_q$ for $q \in \Q \cap [0, 1]$ with $C_0 \subseteq U_q \subseteq \bar{U}_q \subseteq U_r \subseteq \bar{U_r} \subseteq C_1^c$ for $q < r$. Then take $f(x) := \inf \{q \in \Q \cap [0, 1] \ | \ x \in U_q \} \cup \{1\}$.

\subsection{Tietze's extension theorem}
For closed $C$ in a T4 space $X$ and continuous $f: C \to \R$ there is a continuous extension $\tilde{f}: X \to \R$.
\paragraph{Proof idea} Prove extension of $f: C \to ]-1,1[$ to $\tilde{f}: X \to ]-1,1[$, then the result follows by using a homeomorphism $]-1,1[ \to \R$.
By Urysohn's Lemma, it suffices to extend $f: C \to [-1, 1]$ to $\tilde{f}: X \to [-1, 1]$. For this, construct a sequence $h_n: X \to (\frac 2 3)^n [-\frac 1 3, \frac 1 3]$ of continuous functions such that $\sum_n h_n$ converges uniformly.

\subsection{Extension of uniformly continuous functions}
Let $S$ be a set in a metric space $M$ and $f: S \to \R$ uniformly continuous. Then $f$ can be continuously extended to $\tilde{f}: M \to \R$.
\paragraph{Proof idea} Use the following result: If $X$ is a topological space and $Y$ is T3, then for $D \subseteq X$ and continuous $f: D \to Y$ we can extend $f$ to $\bar{D} \to Y$ if 
\begin{equation*}
    \forall x \in \partial D \ \exists y \in Y \ \forall (x_i)_{i \in I} \ \text{net in $D$}: \ x_i \to x \ \Rightarrow \ f(x_i) \to y
\end{equation*}
This condition already determines the extension function $\tilde{f}$, and its continuity can be proven by contradiction. Assume a universal net $(x_i)_{i \in I}$ in $\bar{D}$ converges to $x \in \bar{D}$ but not $\tilde{f}(x_i) \to \tilde{f}(x)$. Construct a net $(w_j)_{j \in J}$ in $D$ such that $w_j \to x$ and $\tilde{f}(w_j)$ is outside of the closure of a fixed neighborhood $N$ of $\tilde{f}(x)$. This contradicts the assumption.

\section{Discrete}

\subsection{Gamma Function}
Defined for $\mathbb{C} \setminus -\mathbb{N}$. Possible definitions:
\begin{equation}
    \begin{split}
        \Gamma(z) &:= \int_0^{\infty} t^{z - 1} e^{-t} dt \ \text{ if } \text{Re}(z) > 0\\
        \frac 1 {\Gamma(z)} &= \lim_{n \mapsto \infty} {\ n + z - 1 \choose\ n } n^{1-z}\nonumber
    \end{split}
\end{equation}
We get
\begin{equation}
    \begin{split}
        \Gamma(z + 1) = z\Gamma(z) \nonumber
    \end{split}
\end{equation}

\section{Functional analysis}

\subsection{Minkowski-functional}
For an absorbing set $A \subseteq X$ the functional
\begin{equation}
    p_A: X \to \R, \quad x \mapsto \mathrm{inf}\{t \geq 0 \ | \ x \in tA \} \nonumber
\end{equation} is
\begin{itemize}
    \item subadditive if $A$ is convex
    \item homogenous if $A$ is balanced
    \item point-separating if $A$ is bounded and $X$ Hausdorff
\end{itemize}

\subsection{Kolmogorov's normability criterion}
$X$ is normable, iff an open, bounded, convex set $A \subseteq X$ exists.
\paragraph{Proof idea} Use the Minkowski-functional for $\tilde{A} = A \cap -A$ which is open, nonempty, bounded, convex.

\subsection{Baire's theorem}
$X$ complete and metric, $\left( A_n \right)_n$ open and dense $ \ \Rightarrow \ \bigcap A_n $ is dense.
\paragraph{Proof idea} For each $y \in X$, construct sequence $(x_n)_n$ with 
\begin{equation}
    x_n \in B_{\frac 1 n}(y) \cap \ \left( \bigcap_{k \leq n} A_n \right) \ \Rightarrow \ y = \lim x_n \in \mathrm{cl}\left( \bigcap_{i \leq k} A_i \right) \text{ for all } k \nonumber
\end{equation}

\subsection{Open mapping theorem}
$X, Y$ Banach and $T: X \to Y$ linear, continuous and surjective. Then $T$ is open.
\paragraph{Proof idea}
\begin{equation}
    \bigcup_{K \in \N} \mathrm{cl}\left(T(B_K(0))\right) = Y \ \Rightarrow \ \mathrm{cl}\left(T(B_K(0))\right)^\circ \neq \emptyset \ \text{for some } K\nonumber
\end{equation}
by Baire's theorem. It follows that $B_\epsilon(0) \subseteq T(B_1(0))$, so $T$ is open, by the following lemma:
\subsubsection{Lemma}
\label{closure_image_unit_ball}
Let $T \in \mathcal{L}(X, Y)$ such that $0 \in \mathrm{cl}(T(B_X))^\circ \neq \emptyset$. Then $0 \in T(B_X)^\circ$, where $B_X = B_1(0)$ is the unit ball.
\paragraph{Proof} The idea is, that $T$ is linear and continuous, so we can work with series. Let $y \in \epsilon B_Y \subseteq \mathrm{cl}(T(B_X))$. Recursivly construct sequences $(x_n)_{n \in \N}$ in $X$ and $(y_n)_{n \in \N}$ in $Y$ with
\begin{align*}
    y_0 = y, &\quad \| y_n \| < 2^{-n}\epsilon, \\
    \| x_n \| < 2^{-n}, &\quad \| y_n - T(x_n) \| < 2^{-n-1} \epsilon \\
    y_{n + 1} &= y_n - T(x_n)
\end{align*}
This is possible as $T(2^{-n}B_X)$ is dense in $2^{-n}\epsilon B_Y$ for each $n \in \N$. By completeness of $X$ have then that $\sum_n x_n$ converges to $x \in X$. Therefore, $T(x) = \sum_n T(x_n) = \sum_n y_n - y_{n + 1} = y_0 = y$ as $y_n \to 0$ for $n \to 0$.

\subsection{Hahn-Banach dominated extension theorem}
Let $X$ be a $\R$-vector space, $p: X \to \R$ sublinear (i.e. subadditive and homogenous w.r.t $\lambda \geq 0$) and $Y \subseteq X$ a subspace. A form $f: Y \to \R$ with $f \leq p$ can be extended to $F: X \to \R$ with $F \leq p$.
\paragraph{Proof idea} Let $F: U \to \R$ be the maximal element (exists by Zorn's lemma) in \begin{equation}
    \left\{ F: U \to \R \ | \ Y \subseteq U \subseteq X, \ \restr{F}{Y} = f, \ F \leq p \right\}  \nonumber
\end{equation}
Then $U = X$, as for $v \in X \setminus U$ have $p(v + y) - F(y) \geq \lambda \geq F(z) - p(z - v)$ for $y, z \in U$ by the reverse triangle inequality. Then $F'(u + tv) := F(u) + \lambda t$ is greater than $F$.

\subsection{Banach-Alaoglu}
$V \subseteq X$ neighborhood of $0 \ \Rightarrow \ K = \{ \phi \in X' \ | \ |\phi(V)| \leq 1 \}$ compact w.r.t weak-*-topology (weakest topology on $X'$ so that all $\hat{x} \in X''$ are continuous, $\hat{x}: X' \to \mathbb{K}, \ \phi \mapsto \phi(x)$).
\paragraph{Proof idea} Let $\gamma(x) > 0$ with $x \in \gamma(x)V$ for all $x \in X$. Then
\begin{equation}
    \mathbb{K}^X = \bigtimes_{x \in X} \mathbb{K} \ \Rightarrow \ K \subseteq \bigtimes_{x \in X} B_{\gamma(x)}(0) \ \text{compact by Tychonoff's theorem}\nonumber
\end{equation}
The topologies on the sets match, as the weak-*-topology on $K$ has a local base of finite intersections of $\hat{x_i}^{-1}(]-\epsilon_i, \epsilon_i[)$ and 
\begin{equation}
    \bigtimes_{x \in X} B_{\gamma(x)}(0) \cap X' \ \text{has one of sets} \ \bigcap_{1 \leq i \leq n} ]-\epsilon_i, \epsilon_i[ \times \bigtimes_{x \neq x_i} \mathbb{K} \cap X' \nonumber
\end{equation}

\section{Operator theory}

\subsection{Neumann series}
Let $T \in \mathcal{L}(X)$. If $\sum_{n \in \N} T^n$ converges, then $1 - T$ is invertible with
\begin{equation*}
    (1 - T)^{-1} = \sum_{n \in \N} T^n
\end{equation*}
To get convergence, it is sufficient to have $\|T\| < 1$ and $X$ is complete.

\subsection{$l^p$ spaces}
Note that from \ref{hoelder_inequality} we get that $l^p \simeq (l^q)'$ for $p > 1$ and $\frac 1 p + \frac 1 q = 1$. 

\subsection{Riesz lemma}
\label{riesz_lemma}
Let $U \subsetneq$ closed subspace of a normed space. For $\delta > 0$ have then $x \in X$ with $\|x\|=1$ and distance greater than $1 - \delta$ from $U$.
\paragraph{Proof idea} Consider any $x \in X \setminus U$ and an almost closest point $u \in U$. Then scale $x - u$ appropriately.

\subsection{Compact Operators and spaces}
From \ref{riesz_lemma} one can conclude that the unit ball $B_X$ is compact iff $\dim X < \infty$. Therefore, consider operators $T \in \mathcal{L}(X, Y)$ such that $\mathrm{cl}(T(B_X))$ compact, these are a Banach space $\mathcal{K}(X, Y)$.
\paragraph{Proof idea} To show that $\mathcal{K}(X, Y)$ is closed in $\mathcal{L}(X, Y)$, consider diagonal sequences.

\subsection{Arzela-Ascoli}
\label{arzela_ascoli}
Let $X$ be a compact topological space. Then the continuous functions $C(X)$ from $X$ to $\R$ are normed via $\|\cdot\|_{\infty}$.
If a $M \subseteq C(X)$ is bounded, closed and equicontinuous (i.e. $\forall x \in X, \epsilon > 0 \ \exists \mathrm{neighborhood} \ N \ \mathrm{of} \ x \ \forall x \in M: \ x(N) \subseteq B_{\epsilon}(x(s))$), then $M$ is compact.

\paragraph{Proof} Let $(x_n)_{n \in \N}$ be a sequence in $M$. As $X$ is compact, it is separable, so have $X = \{ s_n \ | \ n \in \N \}$.
Therefore, recursivly construct subsequences
\begin{equation*}
    \left(x_n^{(k)}\right)_{n \in \N} \ \text{such that} \ \left(x_n^{(k)}(s_k)\right)_{n \in \N} \ \text{converges}
\end{equation*}
and consider the diagonal sequence $(y_n)_{n \in \N}$. Then $(y_n(s_k))_{n \in \N}$ converges for each $k \in \N$.

By equicontinuity, have for each $k \in \N$ a neighborhood $N_k$ of $s_k$ such that $\forall x \in M: \ x(N_k) \subseteq B_\epsilon(x(s_k))$.
Therefore, there is a subcover $N_i$ for $i \in I$ finite. As $(y_n(s_k))_{n \in \N}$ converges for each $k$, it simultaneously converges for each $i \in I$. This yields that $(y_n)_{n \in \N}$ is a Cauchy-sequence w.r.t $\|\cdot\|_{\infty}$.

\subsection{Proposition of Schauder}
For $T \in \mathcal{L}(X, Y)$ between Banach-spaces, have that $T$ is compact if and only if $T' \in \mathcal{L}(Y', X')$ is compact.

\paragraph{Proof} Prove $\Rightarrow$, the other direction follows. Then $K := \mathrm{cl}(T(B_X))$ is compact metric space.
For $(y'_n)_{n \in \N}$ have
\begin{equation*}
    \left(\restr{y'_n}{K}\right)_{n \in \N} \ \text{is a sequence in $C(K)$}
\end{equation*}
It also fulfills the conditions of \ref{arzela_ascoli}, so there is a convergent subsequence indexed by $(n_k)_{k \in \N}$. Then also $(T'y_{n_k})_{k \in \N}$ converges, so $T'(B_{Y'})$ is relatively compact.

\subsection{Closed range theorem}
Let $X, Y$ be Banach spaces, $T \in \mathcal{L}(X, Y)$. The the following are equivalent
\begin{itemize}
    \item $\mathrm{ran}(T)$ closed
    \item $\mathrm{ran}(T) = (\mathrm{ker}(T'))_\perp$
    \item $\mathrm{ran}(T')$ closed
    \item $\mathrm{ran}(T') = (\mathrm{ker}(T))^\perp$
\end{itemize} 
\paragraph{Proof} Show (ii) $\Leftrightarrow$ (iv), the rest is relativly easy. Let $x' \in (\mathrm{ker}(T))^\perp$. Then have $z': \mathrm{ran}(T) \to \mathbb{K}$ linear with $z' \circ T = x'$ (isomorphism theorem).
A complex computation using the open mapping theorem shows that $z'$ is continuous.
A Hahn-Banach extension of $z'$ to $Y$ then yields a preimage under $T'$ of $x'$.

For the other direction, consider $Z := \mathrm{cl}(\mathrm{ran}(T))$. By the Hahn-Banach theorem, we can extend functionals on $Z$ to functionals on $Y$, so $\mathrm{ran}(T') \simeq Z'$ by the isomorphism $\mathrm{ran}(T') \to Z', \ T'(y') \mapsto \restr{y'}{Z}$.

Therefore, for all $y' \in Y'$ have that $\| \restr{y'}{Z} \| \leq c \| y' \circ T \|$ where $c > 0$. 

Consider any $y \in Z$ with $\|y\| \leq 1$. If $y \notin \mathrm{cl}(T(2c B_X))$, the Hahn-Banach separation theorem yields $y' \in Y'$ such that
\begin{equation*}
    2c \| y' \circ T \| = \sup \left( 2c (y' \circ T)(B_X) \right) \leq y'(y) = \| \restr{y'}{Z}(y) \| \leq \| \restr{y'}{Z} \| \leq c \| y' \circ T \| 
\end{equation*}
a contradiction. Therefore, $\mathrm{cl}(T(B_X))^\circ \neq \emptyset$ and so $\tilde{T}: X \to Z, \ x \mapsto T(x)$ is open by \ref{closure_image_unit_ball}.
It follows that $\mathrm{ran}(T) = \mathrm{ran}(\tilde{T})$ is closed, as $X$ is closed.

\subsection{Projection theorem}
\label{projection_theorem}
Let $H$ be a Hilbert space and $K \subseteq H$ convex and closed. Then for $x \in H$ the infimum $\inf_{y \in K} \| y - x\|$ is reached by some $y \in K$. In particular, for $U \subseteq H$ closed subspace, $U^\perp$ is also closed and $H = U \oplus U^\perp$ is a topological decomposition.

\subsection{Frechet-Riesz representation theorem}
Let $H$ be a Hilbert space. Then a isometric, bijective, conjugate linear map is given by
\begin{equation*}
    \phi: H \to H', \quad y \mapsto \langle \cdot , y \rangle
\end{equation*}
\paragraph{Proof} Show surjectivity, the rest is clear: For $x' \in H'$ have that $(\mathrm{ker}(x'))^\perp$ has dimension 1. By using \ref{projection_theorem} the claim follows.

\subsection{Orthonormal bases}
Let $H$ be a Hilbert space and $S \subseteq H$ a maximal orthonormal system. As
\begin{equation*}
    \left\langle x - \sum_{s \in F} \langle x, s \rangle s, \ x - \sum_{s \in F} \langle x, s \rangle s \right\rangle \geq 0 \ \Rightarrow \ \sum_{s \in F} |\langle x, s \rangle|^2 \leq \langle x, x \rangle
\end{equation*}
for finite $F \subseteq S$, get that $\sum_{s \in S} \langle x, s \rangle s$ converges absolutely, and if $x \in \mathrm{cl}(\mathrm{span}(S))$, to $x$. For a maximal orthonormal system $S \subseteq H$ have that $\mathrm{cl}(\mathrm{span}(S)) = H$, so it is an orthonormal basis.

\section{(Algebraic) Number Theory}

\subsection{Propositions (from Neukirch)}
Let $K|\Q$ separable and $\mathcal{O}_K$ integral closure of $\Z$.
\begin{description}
    \item[2.9] For $\alpha_1, ..., \alpha_n \in \mathcal{O}_K$ basis of $K$, then $d(\alpha_1, ..., \alpha_n)\mathcal{O}_K \subseteq \alpha_1 \Z + ... + \alpha_n \Z$.
    \item[2.10] Each finitely generated $\mathcal{O}_K$-module $M \subseteq K$ is a free $\Z$-module. 
    \item[3.1] $\mathcal{O}_K$ is a Dedekind domain, so noetherian, integrally closed and each prime ideal $p \neq 0$ is maximal.
    \item[3.3] Each ideal except $(0), (1)$ has a unique factorization in prime ideals (up to order). 
\end{description}

\subsection{Minkowski's theorem (Neukirch 4.4)}
Let $V$ be a $n$-dimensional euclidean vector space, $\Gamma \subseteq V$ be a complete lattice, $X \subseteq V$ convex and balanced with $\mathrm{vol}(X) > 2^n \mathrm{vol}(\Gamma)$, then $X \cap \Gamma \neq \emptyset$.

\subsection{Dirichlet's unit theorem}
For $K / \Q$ finite with ring of integers $\mathcal{O}_K$, have $\mathcal{O}_K^* \cong \mu(K) \oplus G$, where $\mu(K)$ are the roots of unity and $G$ is a free group of rank $r + s - 1$, where $r$ is the number of real $\Q$-embeddings $K \to \R$ and $s$ is the number of conjugate pairs of complex $\Q$-embeddings $K \to \C$.

\subsection{Square number fields}
For a square-free $D \in \Z, \ D \neq 0, 1$ have $K = \Q(\sqrt{D})$. Then $d := d_K = D$ if $D \equiv 1 \mod 4$ and $d := d_K = 4D$ otherwise. Furthermore, $\mathcal{O}_K = \Z[\frac 1 2 (d_K + \sqrt{d_K})]$.

In the case $D > 1$, have that $\mathcal{O}_K^* = \langle \epsilon_1 \rangle$, where $\epsilon_1 = \frac 1 2 (x + y \sqrt{d_K})$ for the smallest solution $x, y \geq 0$ of $x^2 - dy^2 = -4$ (or $... = 4$ if this has no integral solution).

In the case $D < 0$, have that 
\begin{equation}
    \mathcal{O}_K^* = \begin{cases}
        \{ 1, -1, i, -i \} & \text{if} \ D = -1 \\
        \left\{ e^{\frac {2\pi i k} 6} \middle| k \in \{0, ..., 5\} \right\} & \text{if} \ D = -3 \\
        \{ 1, -1 \} & \text{otherwise}
    \end{cases} \nonumber
\end{equation}
\paragraph{Proof idea of the second part} For $\epsilon = \frac 1 2 (u + v \sqrt{d_K}) \in \mathcal{O}_K^*$ have
\begin{equation*}
    N_{K|\Q}(\epsilon) = \frac 1 4 (u^2 - d_K v^2) \in \{ -1, 1 \} \ \Rightarrow \ u^2 - d_K v^2 = \pm 4
\end{equation*}
By Dirichlet's unit theorem have fundamental unit $\epsilon = \frac 1 2 (u + v \sqrt{d_K})$ and as $-\epsilon$ and $\epsilon^{-1}$ together with $-1$ also generate $\mathcal{O}_K^*$, we may assume $u, v \geq 0$.
Therefore, $\epsilon^k = \frac 1 2 (x + y\sqrt{d_K})$ and as in
\begin{equation*}
    \frac 1 2 (w + t\sqrt{d_K})\frac 1 2 (u + v \sqrt{d_K}) = \frac 1 4 (wu + d_K tv + (ut + vw)\sqrt{d_K})
\end{equation*}
the part $\frac 1 4 (wu + d_K tv)$ is greater than $\frac 1 2 w$ as wlog $u \geq 2$, have that $u, v$ must be the smallest solution of Pell's equation.

\subsection{Ramification (de: Verzweigung)}
Let $\mathcal{R}$ be a Dedekind domain, $K = \mathrm{Quot}(\mathcal{R})$ and $\mathcal{O}$ the integral closure of $\mathcal{R}$ in an algebraic field extension $L|K$. Then $\mathcal{O}$ is a Dedekind domain.

For a prime ideal $\mathfrak{p}$ in $\mathcal{R}$, have
\begin{description}
    \item[8.2] $L|K$ separable $\Rightarrow \ \sum e_i f_i = n := [L : K]$ where $\mathfrak{p}\mathcal{O} = \mathfrak{B}_1^{e_1}...\mathfrak{B}_r^{e_r}$ is the factorization of $\mathfrak{p}$ into prime ideals in $\mathcal{O}$ and $f_i = [\mathcal{O}/\mathfrak{B}_i : \mathcal{R}/\mathfrak{p}]$. The proof uses the CRT and the properties of $\mathcal{O}/\mathfrak{B}_i$ as $\mathcal{R}/\mathfrak{p}$-vector space.
    \item[8.3] Let $L = K(\alpha)$ for an integral, primitive element $\alpha \in \mathcal{O}$. Then $\mathfrak{p} = \mathfrak{B}_1^{e_1}...\mathfrak{B}_r^{e_r}$ for $\mathfrak{B}_i = \mathfrak{p}\mathcal{O} + p_i(\alpha)\mathcal{O}$, where the minimal polynomial $p$ of $\alpha$ splits into irreducible factors mod $\mathfrak{p}\mathcal{O}$ 
    \begin{equation}
        p(X) \equiv p_1(X)^{e_1} ... p_r(X)^{e_r} \mod \mathfrak{p}\mathcal{O} \nonumber
    \end{equation} 
    Also have $f_i = \mathrm{deg}(p_i)$
\end{description}

\end{document}