\documentclass{scrartcl}

\usepackage{graphicx}
\usepackage[utf8]{inputenc}
\usepackage[T1]{fontenc}
\usepackage{lmodern}
\usepackage[ngerman]{babel}
\usepackage{amsmath}
\usepackage{mathtools}
\usepackage{amssymb}
\usepackage{listings}
\usepackage{xparse}
\usepackage{geometry}
\usepackage{hyperref}

\title{Structure-Zoo}
\date{}

\newcommand{\Ex}{\mathrm{E}}
\newcommand{\R}{\mathbb{R}}
\newcommand{\N}{\mathbb{N}}
\newcommand{\Z}{\mathbb{Z}}
\newcommand{\Q}{\mathbb{Q}}
\newcommand{\C}{\mathbb{C}}
\newcommand{\F}{\mathbb{F}}
\newcommand{\vol}{\mathrm{vol}}
\newcommand\restr[2]{{
    \left.\kern-\nulldelimiterspace
    #1
    \vphantom{\big|}
    \right|_{#2}
}}

\begin{document}

\maketitle
\tableofcontents

\section{Topological spaces}

\subsection{Euclidean space}
\label{top:euclidean}
\begin{tabular}{c | p{0.8\textwidth}}
    Definition & $(\R^n, \tau_{\mathrm{Eucl}})$ \\
    \hline
    type & Normed space via $\|\cdot\|_2$ \\
    separation & T4 \\
    compact & no, $(ne_1)_{n \in \N}$ has no convergent subnet \\
    Baire & yes, as completly metrizable \\
    countability & second countable via $\{B_q(r) \ | \ q, r \in \Q, \ q > 0\}$; separable
\end{tabular}
\\\\
Note that this is homeomorphic to each open ball.

\subsection{Compact Euclidean spaces}

\subsection{Euclidean space}
\label{top:compact_euclidean}
\begin{tabular}{c | l}
    Definition & $([0, 1]^n, \restr{\tau_{\mathrm{Eucl}}}{[0, 1]^n})$ \\
    \hline
    type & Normed space via $\|\cdot\|_2$ \\
    separation & T4 \\
    compact & yes, by Heine-Borel \\
    Baire & yes, as completly metrizable \\
    countability & second countable; separable (see \ref{top:euclidean})
\end{tabular}

\subsection{Co-finite topology}
\label{top:co_finite}
Let $X$ be infinite.
\\\\
\begin{tabular}{c | p{0.8\textwidth}}
    Definition & $(X, \tau_{\mathrm{co-finite}} := \{U \subseteq X \ | \ \mathrm{card}(U^c) < \aleph_0 \} \cup \emptyset)$ \\
    \hline
    type & not metric, as not Hausdorff \\
    separation & T1, as there are no disjoint open sets except $\emptyset$ \\
    compact & yes, as a chain of finite sets contains a smallest element \\
    Baire & iff $X$ is uncountable \\
    countability & separable; if $X$ is uncountable then not first countable
\end{tabular}
\\\\
Note that each sequence converges to each point. If $X$ is uncountable, the intersection of countably many open sets is co-countable, so nonempty (this proves the non-first-countability).

Also, each infinite set is dense. If $X$ is countable, therefore all $\{x\}^c$ are open dense but have empty intersection. On the other hand, if $X$ is uncountable, the intersection of countably many nonempty open sets has at most countable complement, so is dense. 

\subsection{Co-countable topology}
\label{top:co_countable}
Let $X$ be uncountable.
\\\\
\begin{tabular}{c | p{0.8\textwidth}}
    Definition & $(X, \tau_{\mathrm{co-countable}} := \{U \subseteq X \ | \ \mathrm{card}(U^c) \leq \aleph_0 \} \cup \emptyset)$ \\
    \hline
    type & not metric, as not Hausdorff \\
    separation & T1, as there are no disjoint open sets except $\emptyset$ \\
    compact & no, as $(\{n, n+1, ...\})_{n \in \N}$ are closed, nonempty with empty intersection \\
    Baire & yes \\
    countability & not separable; not first countable (see \ref{top:co_finite})
\end{tabular}
\\\\
The intersection of countably many nonempty open sets has countable complement, so is open and dense.

\subsection{Discrete topology}
\label{top:discrete}
Let $X$ be a set of at least two elements.
\\\\
\begin{tabular}{c | p{0.8\textwidth}}
    Definition & $(X, 2^X)$ \\
    \hline
    type & metric by $d(x,y) = 1$ if $x \neq y$ \\
    separation & T4 \\
    compact & iff $X$ is finite \\
    Baire & yes, as the only dense set is $X$ \\
    countability & first countable; not separable resp. second countable if $X$ is infinite
\end{tabular}

\subsection{Indiscrete topology}
\label{top:discrete}
Let $X$ be a set of at least two elements.
\\\\
\begin{tabular}{c | p{0.8\textwidth}}
    Definition & $(X, \{\emptyset, X\})$ \\
    \hline
    type & not metric, as not Hausdorff \\
    separation & none \\
    compact & yes \\
    Baire & yes, as the only open, dense set is $X$ \\
    countability & second countable; separable
\end{tabular}

\subsection{Order topology}
\label{top:order_topology}
Let $(X, \leq)$ be a totally ordered set.
\\\\
\begin{tabular}{c | p{0.8\textwidth}}
    Definition & $(X, \tau_{\mathrm{Ord}}$ generated by $X_{<x}$ and $X_{>x}$ for each $x \in X$ \\
    \hline
    type & in general not metric, as not first countable \\
    separation & T4 \\
    compact & iff $\leq$ is order complete (i.e. $\sup$ and $\inf$ exist for all subsets) \\
    Baire & not in general, see $\Q$ \\
    countability & in general neither separable nor first countable, see e.g. $\aleph_1 + 1$
\end{tabular}
\\\\
For $x \notin C$ and $C \subseteq X$ closed have $x \in ]y, z[$ and $]y, z[$ disjoint to $C$ by closedness of $C$. Then one easily sees that $x$ can be separated from $C$ by distinguishing the cases $y < u < x$ or $]y, x[ = \emptyset$ and similarly for $z$.

For T4 see \href{https://math.stackexchange.com/questions/980584/is-every-linear-ordered-set-normal-in-its-order-topology}{math SE}.

\subsubsection{Order topology $\aleph_1 + 1$}

\begin{tabular}{c | p{0.8\textwidth}}
    Definition & $(\aleph_1 + 1, \tau_{\mathrm{Ord}}$ \\
    \hline
    type & not metric, as not first countable \\
    separation & T4 \\
    compact & yes, as order complete \\
    Baire & yes, as compact Hausdorff space \\
    countability & not first countable ($\aleph_1$ has no countable neighborhood basis), not separable (countable union of countable sets is countable)
\end{tabular}

\subsubsection{Order topology $\Q$}

\begin{tabular}{c | p{0.8\textwidth}}
    Definition & $(\Q, \tau_{\mathrm{Ord}}$ \\
    \hline
    type & metric, as subspace topology of $\tau_{\mathrm{Eucl}}$ \\
    separation & T4 \\
    compact & no, as not Baire \\
    Baire & no, as there is no open singleton and $\Q$ is countable \\
    countability & second countable, as $\Q$ is countable
\end{tabular}
\\\\
This is equal to the euclidean subspace topology on $\Q$ 

\subsubsection{Order topology $\aleph_\alpha \times \R$ (lexiographic order)}

\begin{tabular}{c | p{0.8\textwidth}}
    Definition & $(\mathrm{card}(2^\R) \times \R, \tau_{\mathrm{Ord}}$ \\
    \hline
    type & metric, via $d((\mu, x), (\mu, y)) = \frac{d(x, y)}{1 + d(x, y)}$ and $d(u, v) = 1$ otherwise \\
    separation & T4 \\
    compact & no, as $\R$ is not compact \\
    Baire & yes because it is metric and complete \\
    countability & first countable (for $(\mu, x)$ take $(\mu, ]x - \frac 1 n, x + \frac 1 n[)$); second countable and separable iff $\alpha$ is countable
\end{tabular}

\subsection{Infinite-dimensional hypercubes}
\label{top:hypercube}
Let $\alpha$ be a ordinal.
\\\\
\begin{tabular}{c | p{0.8\textwidth}}
    Definition & $([0, 1]^{\aleph_\alpha}, \tau)$ with product topology $\tau$ of $\restr{\tau_{\mathrm{Eucl}}}{[0, 1]}$ \\
    \hline
    type & metric iff $\alpha = 0$, so $\alpha_\alpha$ is countable (with $d(x,y) = \sum_n 2^{-n}|x_n - y_n|$) \\
    separation & T4 \\
    compact & yes by Tychonoffs theorem \\
    Baire & yes, as compact Hausdorff space \\
    countability & first countable iff $\alpha = 0$; In this case, also second countable, separable (see e.g. exercise problem 33); ? separable in the general case ?
\end{tabular}
\\\\
To prove non-first-countability for $\alpha > 0$ use that the sets $]\frac 1 3, \frac 2 3[ \times \prod_{\xi \neq \chi} [0, 1]$ are open and that the intersection of countably many open sets contains $\prod_{\xi \in A} S_\xi \times \prod_{\xi \notin A} [0, 1]$ for a countable $A$ and any $S_\xi \subseteq [0, 1]$.

For the T4 property, consider $C_0, C_1 \subseteq [0,1]^{\aleph_\alpha}$ closed. Then as above $C_i \subseteq A_i \times \prod_{\alpha \notin \mathcal{F}_i} [0,1]$ for $\mathcal{F}_i$ finite. Therefore, separating $C_0, C_1$ at the coordinates $\mathcal{F}_0 \cup \mathcal{F}_1$ is sufficient (this is possible, as the set is finite).

% A countable, dense subset can be constructed as follows: Construct a sequence $(S_\xi)_{\xi < \aleph_\alpha}$ such that each $S_\xi$ is a countable set of partial functions $f: D_f \subseteq \xi \to \Q$ and it holds: For each $\mathcal{F} \subseteq \xi$ finite, $g: \mathcal{F} \to \Q$ there is some $f \in S_\xi$ with $\mathcal{F} \subseteq D_f$ and $g = \restr{f}{\mathcal{F}}$. In the successor ordinal case, just consider each $f \in S_{\xi - 1}$ and extended it by each $q \in \Q$ ($\Q \times S_{\xi - 1}$ is countable). In the limit ordinal case, just unify all previous sets (this does not work).

By Urysohn's Lemma, each compact T4 space is homeomorphic to a subspace of $[0, 1]^\kappa$, where $\kappa$ is a big enough cardinal number. It suffices to take $\kappa = \mathrm{card}(X^2)$, but it even suffices if there is a base of cardinality $\kappa$. Note that because of this, every second countable compact T4 space is metrizable.

\subsection{Sorgenfrey topology}
\label{top:sorgenfrey}
\begin{tabular}{c | p{0.8\textwidth}}
    Definition & $(\R, \tau_{\mathrm{Sorg}})$ where $\tau_{\mathrm{Sorg}}$ is generated by $[a, b[$ \\
    \hline
    type & not metric, as separable but not first countable \\
    separation & T4 \\
    compact & no, even the subspace $[0, 1]$ is not compact, as the cover $[1, 2[$ and $[1 - \frac 1 n, 1 - \frac 1 {n + 1}[$ has no finite subcover \\
    Baire & yes \\
    countability & first countable, separable; not second countable
\end{tabular}
\\\\
Note that the Sorgenfrey is not connected. Assume there is a countable basis. Then there is some $x$ such that $x$ is not the infimum of any basis set. Then there is no basis set $B$ with $x \in B \subseteq [x, x + \epsilon]$.

To see that it is a Baire space, consider $(U_n)_{n \in \N}$ open, dense and construct decreasing $[x_n, x_n + \epsilon_n[$ in $U_n \cap N$ for some open $N$. Then the $x_n$ have a supremum $x$, and it is in $\bigcap_{n \in \N} U_n$.
\end{document}