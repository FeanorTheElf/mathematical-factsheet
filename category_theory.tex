
\subsection{Presheaves and Sheaves}
Let $\mathcal{C}$ be a small category and $\mathcal{D}$ a sub-category of Set.
Then the category of presheaves $\mathrm{PreSh}(\mathcal{C}, \mathcal{D})$ is the category of functors $\mathcal{C} \to \mathcal{D}$, i.e.
\begin{equation*}
    \mathrm{Ob}(\mathrm{PreSh}(\mathcal{C}, \mathcal{D})) = \{ F: \mathcal{C} \to \mathcal{D} \ \text{functor} \}, \quad \mathrm{Hom}_{\mathrm{PreSh}(\mathcal{C}, \mathcal{D})}(F, G) = \{ \eta: F \Rightarrow G \}
\end{equation*}
We also write $\mathrm{PreSh}(\mathcal{C})$ for $\mathrm{PreSh}(\mathcal{C}, \mathrm{Set})$.
If $X$ is a topological space, consider the category $\mathrm{Top}(X)$ defined by
\begin{equation*}
    \mathrm{Ob}(\mathrm{Top}(X)) = \{ U \subseteq X \ \text{open} \}, \quad \mathrm{Hom}_{\mathrm{Top}(X)}(U, V) := \begin{cases}
        \{ U \to V \} & \text{if $U \subseteq V$} \\
        \emptyset & \text{otherwise}
    \end{cases}
\end{equation*}
We also write $\mathrm{PreSh}(X, \mathcal{D})$ for $\mathrm{PreSh}(\mathrm{Top}(X)^{\mathrm{op}}, \mathcal{D})$.

Finally, for a topological space $X$ and a sub-category $\mathcal{D}$ of Set, we define the category of sheaves $\mathrm{Sh}(X, \mathcal{D})$ by
\begin{equation*}
    \mathrm{Ob}(\mathrm{Sh}(X, \mathcal{D})) := \{ F \in \mathrm{Ob}(\mathrm{PreSh}(X, \mathcal{D})) \ | \ F \ \text{satisfies ``local-to-global condition''} \}
\end{equation*}
Here a functor $F$ satisfies the ``local-to-global condition'', if for all families of open sets $(U_i)_{i \in I}$ and elements $s_i \in F(U_i)$ with
\begin{equation*}
    \forall i, j: F(U_i \cap U_j \to U_i)(s_i) = F(U_i \cap U_j \to U_j)(s_j)
\end{equation*}
it holds that there is a unique $s \in F(\bigcup_j U_j)$ with
\begin{equation*}
    \forall i: F\biggl(U_i \to \bigcup_j U_j\biggr)(s) = s_i
\end{equation*}

\subsection{Sheafification}
Let $F$ be a presheaf on a topological space $X$ and a sub-category $\mathcal{D}$ of Set that has small limits and colimits.
Then for $x \in X$, define the stalk of $F$ at $x$ as
\begin{equation*}
    F_x := \colim_{U \ni x} F(U)
\end{equation*}
Further define a functor $F^+: \mathrm{Top}(X)^{\mathrm{op}} \to \mathcal{D}$ by
\begin{align*}
    &F^+(U) := \bigl\{ \ s \in \prod_{x \in U} F_x \ \bigm| \ \forall u \in U \ \exists V \subseteq U \ \text{open}, \ \sigma \in F(V): \ u \in V, \ \underbrace{s_v = \sigma \in F_v}_{\mathclap{\substack{\text{strictly speaking, the images under the}\\\text{maps $F(U), F(V) \to F_v$ must be equal}}}} \ \bigr\} \\
    &F^+(U \to V): \prod_{u \in V} F_u \to \prod_{u \in U} F_u, \quad (s_u)_{u \in V} \mapsto (s_u)_{u \in U}
\end{align*}
Then $F^+$ is a sheaf on $X$ and $\mathcal{D}$ and there is a natural transformation $\eta: F \Rightarrow F^+, \eta_U: s \to (s \in F_u)_{u \in U}$ such that every natural transformation $\theta: F \Rightarrow G$ factors as
\begin{equation*}
    F \overset{\eta}{\Rightarrow} F^+ \overset{\theta'}{\Rightarrow} G
\end{equation*}
with an unique $\theta': F^+ \Rightarrow G$.
Furthermore, this is a universal property of $F^+$.

\subsection{Yoneda's Lemma}
For a small category $\mathcal{C}$, Yoneda's functor
\begin{align*}
    Y: \mathcal{C} \to \mathrm{PreSh}(\mathcal{C}) := \mathrm{Fun}(\mathcal{C}^{\mathrm{op}}, \mathrm{Set}), \quad x &\mapsto \mathrm{Hom}_{\mathcal{C}}(\cdot, x) \\
    f \in \mathrm{Hom}_{\mathcal{C}}(x, y) &\mapsto \eta: \mathrm{Hom}(\cdot, x) \Rightarrow \mathrm{Hom}(\cdot, y), \ \eta_z: g \mapsto f \circ g
\end{align*}
is fully faithful.

\subsection{Characterization of adjoint functors}
Let $F: \mathcal{C} \to \mathcal{D}$ and $G: \mathcal{D} \to \mathcal{C}$ be functors between locally small categories.
Then $F \dashv G$ iff there is a collection of isomorphisms
\begin{equation*}
    \phi_{xy}: \mathrm{Hom}_{\mathcal{D}}(F(x), y) \to \mathrm{Hom}_{\mathcal{C}}(x, G(y))
\end{equation*}
that it natural in $x \in \mathrm{Ob}\mathcal{C}$ and $y \in \mathrm{Ob}\mathcal{D}$ in the following sense:
For $\alpha: y_1 \to y_2$ and $\beta: x_1 \to x_2$ have
\begin{equation*}
    \phi_{x_1 y_2} \circ (\alpha \circ \cdot) = (G(\alpha) \circ \cdot) \circ \phi_{x_1 y_1} \quad \text{and} \quad \phi_{x_2 y_1} \circ (\cdot \circ F(\beta)) = (\cdot \circ \beta) \circ \phi_{x_1 y_1}
\end{equation*}
In other words, we have a natural isomorphism
\begin{equation*}
    \mathrm{Hom}_{\mathcal{D}}(F(\cdot), \cdot) \ \Rightarrow \ \mathrm{Hom}_{\mathcal{C}}(\cdot, G(\cdot))
\end{equation*}
where the functors are $\mathcal{C} \times \mathcal{C} \to \mathrm{Set}$.

\subsection{Limits commute}
Let $F: I \times J \to \mathcal{C}$ a diagram and assume $\mathcal{C}$ has limits of shape $I$ and of shape $J$.
Then
\begin{equation*}
    \lim_I \lim_J F \cong \lim_{I \times J} F \cong \lim_J \lim_I F
\end{equation*}
Dually, if $\mathcal{C}$ has colimits of shape $I$, have that
\begin{equation*}
    \colim_I \colim_J F \cong \colim_{I \times J} \cong \colim_J \colim_I F
\end{equation*}

\subsection{Limits and adjoint functors}
Let $F: \mathcal{C} \to \mathcal{D}$ be a functor with right-adjoint $F \dashv G$, then $F$ preserves colimits (and $G$ preserves limits), i.e. if the colimit $\colim_I \Lambda$ exists in $\mathcal{C}$ for some diagram $\Lambda: I \to \mathcal{C}$, then $F(\colim_I \Lambda) \cong \colim_I F \Lambda$ (and the right-hand side exists in $\mathcal{D}$). 
\paragraph{Proof}
It suffices to show that $G(\lim_I \Lambda)$ is a limit of $G \Lambda: I \to \mathcal{C}$.
For $i \in I$ have morphisms $f_i: \lim_I \Lambda \to \Lambda(i)$ and so have morphisms
\begin{equation*}
    G(f_i): G(\lim_I \Lambda) \to G \Lambda(i)
\end{equation*}
which clearly commute in the sense $G(\alpha) \circ G(f_i) = G(f_j)$ where $\alpha: i \to j$.
Hence we only have to show the universal property.

As the Yoneda embedding is fully faithful, we may assume that there is a presheaf $W \in \mathrm{PreSh}(\mathcal{C})$ with morphisms $W \to \mathrm{Hom}(\cdot, G\Lambda(i))$ that are nicely compatible with maps $i \to j$.
Now it suffices to show that the Hom-functor commutes with limits, as then
\begin{equation*}
    \mathrm{Hom}(\cdot, G(\lim_I \Lambda)) \cong \mathrm{Hom}(F(\cdot), \lim_I \Lambda) \cong \lim_{i \in I} \mathrm{Hom}(F(\cdot), \Lambda(i)) \cong \lim_{i \in I} \mathrm{Hom}(\cdot, G\Lambda(i))
\end{equation*}
and the claim follows by the universal property of $\lim_{i \in I} \mathrm{Hom}(\cdot, G\Lambda(i))$.
This is shown in the next lemma.$\hfill\square$

\subsubsection{Lemma (Hom-functor preserves limits)}
For a locally small category $\mathcal{C}$ and a diagram $F: I \to \mathcal{C}$ such that $\lim_I F$ exists, have a natural isomorphism
\begin{equation*}
    \mathrm{Hom}_{\mathcal{C}}(\cdot, \lim_I F) \ \Rightarrow \ \lim_{i \in I} \mathrm{Hom}_{\mathcal{C}}(\cdot, F(i))
\end{equation*}
In other words, there is a bijection $\mathrm{Hom}_{\mathcal{C}}(x, \lim_I F) \cong \lim_{i \in I} \mathrm{Hom}_{\mathcal{C}}(x, F(i))$ that is natural in $x \in \mathrm{Ob}\mathcal{C}$.

Dually find
\begin{equation*}
    \mathrm{Hom}_{\mathcal{C}}(\colim F, \cdot) \ \Rightarrow \ \lim_{i \in I} \mathrm{Hom}_{\mathcal{C}}(F(i), \cdot)
\end{equation*}
\paragraph{Proof}
First we show that $\mathrm{Hom}(x, \lim F) \cong \lim_{i \in I} \mathrm{Hom}(x, F(i))$.
Clearly $\lim_I F$ has morphisms $\lim_I F \to F(i)$ which induce morphisms that are nicely compatible with morphisms $i \to j$. 
So we show the universal property.

Let $W$ be a set with morphisms $g_i: W \to \mathrm{Hom}(x, F(i))$. Then the $g_i$ factor via the unique morphism $\phi: W \to \mathrm{Hom}(x, \lim_I F)$ defined as follows.
Note that for $h \in W$ we have morphisms
\begin{equation*}
    g_i(h): x \to F(i)
\end{equation*}
that are nicely compatible with morphisms $i \to j$.
Hence, we find a unique morphism $x \to \lim_I F$ by the universal property of $\lim_I F$, which shall be $\phi(h)$.

Now the naturality of the constructed isomorphism is easy to show.$\hfill\square$

\subsection{Limit functor adjunction}
Assume $\mathcal{C}$ has limits of shape $I$.
Then the colimit functor
\begin{equation*}
    \colim_I: \mathrm{Fun}(I, \mathcal{C}) \to \mathcal{C}
\end{equation*}
satisfies $\colim_I \dashv \Delta$ where
\begin{equation*}
    \Delta: \mathcal{C} \to \mathrm{Fun}(I, \mathcal{C}), \quad x \mapsto (i \mapsto x)
\end{equation*}
is the diagonal functor.

\subsection{Existence of Limits}
\begin{itemize}
    \item $\mathcal{C}$ has limits, if and only if it has products and equalizers
    \item $\mathcal{C}$ has finite limits, if and only if it has binary products, a final object and equalizers
\end{itemize}

\subsection{Density of representable functors}
Let $\mathcal{C}$ be a locally small category. 
Then for each $\Gamma \in \mathrm{PreSh}(\mathcal{C}) := \mathrm{Fun}(\mathcal{C}^{\mathrm{op}}, \mathrm{Set})$ there exists a category $\epsilon(\mathcal{C})$ and a functor $P_\Gamma: \epsilon(\Gamma) \to \mathrm{PreSh}(\mathcal{C})^{\mathrm{repr}}$ such that
\begin{equation*}
    \colim_{\epsilon(\Gamma)} P_\Gamma = \Gamma
\end{equation*}
\paragraph{Proof} Let
\begin{align*}
    \epsilon(\Gamma) = (* \Rightarrow \Gamma)^{\mathrm{op}} \ \text{so} \ &\mathrm{Ob}\epsilon(\Gamma) = \{ (x, y) \ | \ x \in \mathrm{Ob}\mathcal{C}, y \in \Gamma(x)\} \\
    &\mathrm{Hom}_{\epsilon(\Gamma)}((x_1, y_1), (x_2, y_2)) = \{ f \in \mathrm{Hom}_{\mathcal{C}}(x_1, x_2) \ | \ \Gamma(f)(y_2) = y_1 \}
\end{align*}
and
\begin{align*}
    P_\Gamma: \epsilon(\Gamma) \to \mathrm{PreSh}(\mathcal{C})^{\mathrm{repr}}, \quad (x, y) \mapsto \ &Y(x) = \mathrm{Hom}(\cdot, x) \\
    f \in \mathrm{Hom}_{\epsilon(\Gamma)}((x_1, y_1), (x_2, y_2)) \mapsto \ &\eta^{(f)}: Y(x_1) \Rightarrow Y(x_2) \\
    &\eta^{(f)}_z: \mathrm{Hom}(z, x_1) \to \mathrm{Hom}(z, x_2), \quad f \circ \cdot
\end{align*}
\paragraph{Show} $\Gamma$ is the colimit of $P_\Gamma$.

For all $(x, y) \in \mathrm{Ob}\epsilon(\Gamma)$ consider the natural transformations
\begin{align*}
    \theta^{(x, y)}: P_\Gamma(x, y) \Rightarrow \Gamma, \quad \theta^{(x, y)}_z: \mathrm{Hom}_{\mathcal{C}}(z, x) \to \Gamma(z), \ g \mapsto \underbrace{\Gamma(g)}_{\mathclap{\in \mathrm{Hom}_{\mathrm{Set}}(\Gamma(x), \Gamma(z))}}(y)
\end{align*}
We have for $\alpha: (x_1, y_1) \to (x_2, y_2)$ morphism in $\epsilon(\Gamma)$ that
\begin{equation*}
    \theta^{(x_1, y_1)} = \theta^{(x_2, y_2)} \circ P_\Gamma(\alpha)
\end{equation*}
as for $f \in P_\Gamma(x_1, y_1)(z) = \mathrm{Hom}_{\mathcal{C}}(z, x_1), z \in \mathrm{Ob}\mathcal{C}$ it holds
\begin{align*}
    \theta^{(x_1, y_1)}_z(f) &= \Gamma(f)(y_1) = \Gamma(f)(\Gamma(\alpha)(y_2)) = \Gamma(\alpha \circ f)(y_2) \\
    &= \theta^{(x_2, y_2)}(\alpha \circ f) = \theta^{(x_2, y_2)}_z \circ \eta^{(\alpha)}_z(f) = (\theta^{(x_2, y_2)}_z \circ P_\Gamma(\alpha))(f)
\end{align*} 

So it is left to show the universal property. Let $W \in \mathrm{PreSh}(\mathcal{C})$ with natural transformations
\begin{equation*}
    \phi^{(x, y)}: P_\Gamma(x, y) \Rightarrow W, \quad \phi^{(x, y)}_z: \mathrm{Hom}_{\mathcal{C}}(z, x) \to W(z)
\end{equation*}
such that for all $\alpha: (x_1, y_1) \to (x_2, y_2)$ morphisms in $\epsilon(\Gamma)$ have
\begin{equation*}
    \phi^{(x_1, y_1)} = \phi^{(x_2, y_2)} \circ P_\Gamma(\alpha)
\end{equation*}
Then there exists the unique natural transformation
\begin{equation*}
    \rho: \Gamma \Rightarrow W, \quad \rho_z: \Gamma(z) \to W(z), \ y \mapsto \phi^{(z, y)}_z(\mathrm{id}_z)
\end{equation*}
such that
\begin{equation*}
    \rho \circ \theta^{(x, y)} = \phi^{(x, y)}
\end{equation*}
This is the case, as for $f \in \mathrm{Hom}_{\mathcal{C}}(z, x)$ we find $\Gamma(f)(y) = y' \in \Gamma(z)$.
Note that if follows that $f: (z, y') \to (x, y)$ is a morphism in $\epsilon(\Gamma)$ and so
\begin{equation*}
    \phi^{(z, y')} = \phi^{(x, y)} \circ P_\Gamma(f) = \phi^{(x, y)} \circ \eta^{(f)}
\end{equation*}
Then
\begin{align*}
    \phi^{(x, y)}_z(f) &= \phi^{(x,y)}_z(\eta^{(f)}_z(\mathrm{id}_z)) = (\phi^{(x, y)} \circ \eta^{(f)})_z(\mathrm{id}_z) = \phi^{(z, y')}_z(\mathrm{id}_z) \quad \text{and} \\
    (\rho \circ \theta^{(x, y)})_z(f) &= \rho(\Gamma(f)(y)) = \rho(y') = \phi_z^{(z, y')}(\mathrm{id}_z)
\end{align*}
The uniqueness is clear, and the claim follows.$\hfill\square$

