
\subsection{Parameterized Algorithms design techniques}

\subsubsection*{Kernelization}

Given an instance $(X, k)$, compute an instance $(X', k')$ such that $(X, k)$ is a YES-instance if and only if $(X', k')$ is a YES-instance and the size of $X'$ is bounded by $f(k)$ (this is usually done via reduction rules).

\subsubsection*{Bounded search tree}

Given an instance $(X, k)$ compute instances $(X_1, k_1)$, ..., $(X_n, k_m)$ such that $(X, k)$ is a YES-instance if and only if at least one of the $(X_i, k_i)$ is a YES-instance. Additionally, it must hold that $d_i := k - k_i > 0$.
Then applying this algorithm recursively on all the $(X_i, k_i)$ yields an FPT algorithm with running time $O(\lambda^k \mathrm{poly}(n))$ where $\lambda$ is the positive root of the branching vector polynomial equation (there exactly one, as the $m-1, ..., 0$-th derivates all are non-positive up to some point, and then positive)
\begin{equation*}
    X^m = \sum_{i = 1}^m X^{m - d_i}
\end{equation*}

\subsubsection*{Iterative Compression}

Usually this is used for problems where solutions are (in some way) parts/subsets of the instance and behave monotonously (i.e. supersets of solutions are also solutions).
In this case, one normally has the size of the searched solution as parameter.
Then it is easy to reduce the original problem to its ``compression variant''
\begin{description}
    \item[Input] Given an instance $(X, k)$ and a solution to $Z \leq X$ of size exactly $k + 1$
    \item[Problem] Find a solution of size at most $k$ (or find that it does not exist)
\end{description}
By trying all $2^k$ subsets, we can further reduce it to the ``disjoint variant'' of the problem:
\begin{description}
    \item[Input] Given an instance $(X, k)$ and a solution $Y \leq X$ of size exactly $k + 1$
    \item[Problem] Find a solution of size at most $k$ that is disjoint to $Y$ (or that it does not exist) 
\end{description}

\subsection{Treewidth}

A tree $T$ with nodes in $2^V$ is called a tree decomposition of $G = (V, E)$, if
\begin{itemize}
    \item for $\{u, v\} \in E$ have a tree node containing $u, v$
    \item for $u \in V$ have a tree node containing $u$
    \item for $u \in V$ have that all tree nodes containing $u$ form a subtree of $T$
\end{itemize}
The width of a tree decomposition $T$ is the cardinality of its largest node minus 1.
The treewidth of a graph $G = (V, E)$ is the minimal width of a tree decomposition. 
