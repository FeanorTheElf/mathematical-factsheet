

\subsection{Riesz Representation theorem}
Let $K$ be a compact metric space (compact Hausdorff space??). Consider $(C(X), \|\cdot\|_\infty)$ and the space of complex regular Borel measures $M(X)$ on $X$. Then $C(K)' \cong M(K)$ under
\begin{equation*}
    M(K) \to C(K)', \quad \mu \mapsto \Bigl(f \mapsto \int_K f \ d\mu\Bigr)
\end{equation*}

\subsection{Compact Operators and spaces}
From \ref{riesz_lemma} one can conclude that the unit ball $B_X$ is compact iff $\dim X < \infty$. Therefore, consider operators $T \in \mathcal{L}(X, Y)$ such that $\mathrm{cl}(T(B_X))$ compact, these are a Banach space $\mathcal{K}(X, Y)$.
\paragraph{Proof idea} To show that $\mathcal{K}(X, Y)$ is closed in $\mathcal{L}(X, Y)$, consider diagonal sequences.

\subsubsection{Riesz lemma}
\label{riesz_lemma}
Let $U \subsetneq X$ closed subspace of a normed space. For $\delta > 0$ have then $x \in X$ with $\|x\|=1$ and distance greater than $1 - \delta$ from $U$.
\paragraph{Proof idea} Consider any $x \in X \setminus U$ and an almost closest point $u \in U$. Then scale $x - u$ appropriately.

\subsection{Arzela-Ascoli}
\label{arzela_ascoli}
Let $X$ be a compact metric space. Then the continuous functions $C(X)$ from $X$ to $\R$ are normed via $\|\cdot\|_{\infty}$.
If a $M \subseteq C(X)$ is bounded, closed and equicontinuous (i.e. $\forall x \in X, \epsilon > 0 \ \exists \mathrm{neighborhood} \ N \ \mathrm{of} \ x \ \forall x \in M: \ x(N) \subseteq B_{\epsilon}(x(s))$), then $M$ is compact.

\paragraph{Proof} Let $(x_n)_{n \in \N}$ be a sequence in $M$. As $X$ is compact, it is separable, so have $X = \mathrm{cl}(\{ s_n \ | \ n \in \N \})$.
Therefore, recursivly construct subsequences
\begin{equation*}
    \left(x_n^{(k)}\right)_{n \in \N} \ \text{such that} \ \left(x_n^{(k)}(s_k)\right)_{n \in \N} \ \text{converges}
\end{equation*}
and consider the diagonal sequence $(y_n)_{n \in \N}$. Then $(y_n(s_k))_{n \in \N}$ converges for each $k \in \N$.

By equicontinuity, have for each $k \in \N$ a neighborhood $N_k$ of $s_k$ such that $\forall x \in M: \ x(N_k) \subseteq B_\epsilon(x(s_k))$.
Therefore, there is a subcover $N_i$ for $i \in I$ finite. As $(y_n(s_k))_{n \in \N}$ converges for each $k$, it simultaneously converges for each $i \in I$. This yields that $(y_n)_{n \in \N}$ is a Cauchy-sequence w.r.t $\|\cdot\|_{\infty}$.

\subsection{Projection theorem}
\label{projection_theorem}
Let $H$ be a Hilbert space and $K \subseteq H$ convex and closed. Then for $x \in H$ the infimum $\inf_{y \in K} \| y - x\|$ is reached by some $y \in K$. In particular, for $U \subseteq H$ closed subspace, $U^\perp$ is also closed and $H = U \oplus U^\perp$ is a topological decomposition.
\paragraph{Proof} We have $\| x + y \|^2 + \| x - y \|^2 = 2\| x \|^2 + 2\| y \|^2$. For any sequence $(x_n)_n$ in $K$ that has $\| x_n - x \| \to d := \inf_{y \in K} \| y - x\|$ we then have:
\begin{equation*}
    \frac 1 4 \| x_n - x_m \|^2 \leq \frac 1 2 \| x_n - x \|^2 + \frac 1 2 \| x_m - x \|^2 - \| \underbrace{\frac 1 2 x_n + \frac 1 2 x_m}_{\in K} - x \|^2 
\end{equation*}
If we choose $n, m$ large enough that $\| x_n - x \|^2, \| x_m - x \|^2 \leq d^2 + \epsilon$ then it follows
\begin{equation*}
    \frac 1 4 \| x_n - x_m \|^2 \leq d^2 + \epsilon - d^2 = \epsilon \quad \text{so} \quad \| x_n - x_m \| \leq 4\epsilon
\end{equation*}
So $(x_n)_n$ is a Cauchy sequence and converges to the searched point $y \in K$ (as $K$ is closed).

\subsection{Frechet-Riesz representation theorem}
Let $H$ be a Hilbert space. Then a isometric, bijective, conjugate linear map is given by
\begin{equation*}
    \phi: H \to H', \quad y \mapsto \langle \cdot , y \rangle
\end{equation*}
\paragraph{Proof} Show surjectivity, the rest is clear: For $x' \in H'$ have that $(\mathrm{ker}(x'))^\perp$ has dimension 1. By using \ref{projection_theorem} the claim follows.

\subsection{Spectra}
Let $T \in \mathcal{L}(X)$ for a Banach space $X$.
With
\begin{align*}
    \text{point spectrum} \quad \sigma_p(T) &:= \{ \lambda \in \mathbb{K} \ | \ \ker(T - \lambda) \neq \emptyset \} \\
    \text{continuous spectrum} \quad \sigma_c(T) &:= \{ \lambda \in \mathbb{K} \ | \ \ker(T - \lambda) = \emptyset, \ \mathrm{cl}(\mathrm{im}(T - \lambda)) \neq X \} \\
    \text{residual spectrum} \quad \sigma_r(T) &:= \{ \lambda \in \mathbb{K} \ | \ \ker(T - \lambda) = \emptyset, \ \mathrm{cl}(\mathrm{im}(T - \lambda)) = X, \ \mathrm{im}(T - \lambda) \neq X \} \\
    \text{spectrum} \quad \sigma(T) &:= \sigma_p(T) \cup \sigma_c(T) \cup \sigma_r(T)
\end{align*}
have that $\sigma(T)$ compact and bounded by $\| T \|_{\mathrm{op}}$.
\paragraph{Proof idea} Use the Neumann series $\sum_n T^n = (1 - T)^{-1}$ if convergent.

\subsection{Spectral theorem for compact, normal operators}
\label{spectral_theorem_operator}
Let $T \in \mathcal{K}(H)$ on a Hilbert space $H$ be normal (if $\mathbb{K} = \C$) resp. self-adjoint (if $\mathbb{K} = \R$). Then there is a countable orthonormal system $E$ and $\lambda_e \in \mathbb{K} \setminus \{0\}$ for $e \in E$ such that
\begin{equation*}
    T = \sum_{e \in E} \lambda_e \langle \cdot, e \rangle e
\end{equation*}
Additionally, $\{ \lambda_e \ | \ e \in E \}$ has $0$ as only accumulation point, is bounded by $\| T \|_{\mathrm{op}}$ and $\lambda_e$ takes the same value for only finitely many $e \in E$. Also $H = \ker T \oplus \mathrm{cl}(\mathrm{span}(E))$.
\paragraph{Proof} For $\lambda, \mu \in \sigma(T)$ with $\lambda \neq \mu$ have that $\ker(T - \lambda) \perp \ker(T - \mu)$ as $\mu v = T v = \lambda v$ implies $v = 0$. Therefore, take for $\lambda \in \sigma(T)$ orthonormal basis $\{ e_{\lambda, 1}, ..., e_{\lambda, n_\lambda} \}$ of $\ker(T - \lambda)$ and set
\begin{equation*}
    E = \{ e_{\lambda, i} \ | \ \lambda \in \sigma(T) \setminus \{0\} \}, \quad \lambda_{e_{\lambda, i}} = \lambda
\end{equation*}
Now consider $H_2 := (\ker T + \mathrm{cl}(\mathrm{span}(E)))^\perp$. Then $T(H_2) \subseteq H_2$ and $T_2 := \restr{T}{H_2}: H_2 \to H_2$ is compact and self-adjoint. If $T_2 = 0$ then $\ker(T_2) \subseteq H_2 \cap \ker(T) = \{0\}$ so we are done. 
So assume $T_2 \neq 0$. Then $T_2x = \lambda x$ for some $\lambda \neq 0$ (see next lemma). However, then $x \in \ker(T - \lambda)$, a contradiction.
The rest of the proposition follows from the next lemmas:

\subsubsection{Decomposition compact operator}
\label{decomposition_compact_operator}
Let $T \in \mathcal{K}(X)$ for Banach space $X$. Then $X = \ker((T - 1)^p) \oplus \mathrm{im}((T - 1)^p)$ for some $p \in \N$ (where the direct sum is a decomposition in the topological sense).
\paragraph{Proof idea} Show that the sequence of $N_i = \ker((T - 1)^i)$ is stationary. Assume not, then have $x_i \in N_i$ with $\|x_i\| = 1$ and distance $\frac 1 2$ to $N_{i - 1}$ by Riesz Lemma. Applying $T$ then yields a non-Cauchy sequence as for $m < n$ have
\begin{equation*}
    Tx_n - Tx_m = x_n - x_m + (T - 1)(x_n - x_m) \in x_n - \underbrace{x_m + \ker((T - 1)^{n - 1})}_{= N_{n - 1}}
\end{equation*}
a contradiction to the compactness of $T$. Similar show that $\mathrm{im}((T - 1)^i)$ is stationary and for an index $p \in \N$ at which both are constant the claim holds. The closedness of $\mathrm{im}((T - 1)^p)$ follows as $(T - 1)^p$ is open by the open mapping theorem.$\hfill\square$

\subsubsection{Lemma}
A compact operator $T \in \mathcal{K}(H)$ that is normal (if $\mathbb{K} = \C$) resp. self-adjoint (if $\mathbb{K} = \R$) has $\lambda \in \sigma(T)$ where $|\lambda| = \| T \|_{\mathrm{op}}$.

\subsubsection{Lemma (Spectrum of compact operators)}
Let $T \in \mathcal{K}(X)$. Then $\sigma(T)$ is countable with only accumulation point $0$.
\paragraph{Proof idea} Assume there are infinitely many $\lambda_n \in \sigma(T)$ pairwise distinct with $|\lambda_n| > \epsilon > 0$. By \ref{decomposition_compact_operator} each $T - \lambda_n$ is injective iff surjective, so have $Tx_n = \lambda_n x_n$ for non-zero $x_n$. 
It follows that they are linearly independent. By Riesz lemma, have $y_n \in \mathrm{span}\{x_1, ..., x_n\}$ with distance $\frac 1 2$ to $\mathrm{span}\{x_1, ..., x_{n - 1}\}$ and $\| y_n \| = 1$. Then $Ty_n$ has distance $\frac 1 2 \epsilon$ from $\mathrm{span}\{Tx_1, ..., Tx_{n-1}\}$, but this contradicts the compactness of $T$.

\subsection{Singular value decomposition}
Let $T \in \mathcal{K}(H_1, H_2)$. Then there is $N = \{ 1, ..., n \}$ or $N = \N$ and orthonormal systems $\{ e_n \ | \ n \in N \}$ of $H_1$ and $\{ f_n \ | \ n \in N \}$ of $H_2$ and $\{ s_n \ | \ n \in N \} \subseteq \R_{>0}$ with $0$ as only accumulation point such that
\begin{equation*}
    T = \sum_{n \in N} s_n \langle \cdot, e_n \rangle f_n
\end{equation*}
\paragraph{Proof idea} 
The operator $T^* \circ T$ is positive, self-adjoint and compact, so has a unique positive, self-adjoint compact root $S$ with $S \circ S = T^* \circ T$ (take the root of each eigenvalue in the representation of \ref{spectral_theorem_operator}).
Then $T = U \circ S$ for a unitary operator $U$ and with $S = \sum_{e \in E} \lambda_e \langle \cdot, e \rangle e$ have that
\begin{equation*}
    T = \sum_{e \in E} \lambda_e \langle \cdot, e \rangle Ue
\end{equation*}
which is of the specified form.$\hfill\square$
